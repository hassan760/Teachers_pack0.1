\documentclass[12pt, french]{article}

\usepackage{fancyhdr, fancybox, lastpage}
\usepackage[most]{tcolorbox}
\usepackage[a4paper, margin={0.3in, .75in}]{geometry}
\usepackage{wrapfig}
\pagestyle{fancy}
\usepackage[siunitx, RPvoltages]{circuitikz}
\renewcommand\headrulewidth{1pt}
\renewcommand\footrulewidth{1pt}
\fancyhf{}
\rhead{ \em{Zakaria Haouzan}}
\lhead[C]{\em{1ére Année Baccalauréat Sciences Expérimentales}}
\chead[C]{}
\rfoot[C]{}
\lfoot[R]{}
\cfoot[]{\em{Page \thepage / \pageref{LastPage}}}

\newcommand{\mymotor}[2] % #1 = name , #2 = rotation angle
{\draw[thick,rotate=#2] (#1) circle (10pt)
 node[]{$\mathsf M$} 
++(-12pt,3pt)--++(0,-6pt) --++(2.5pt,0) ++(-2.8pt,6pt)-- ++(2.5pt,0pt);
\draw[thick,rotate=#2] (#1) ++(12pt,3pt)--++(0,-6pt) --++(-2.5pt,0) ++(2.8pt,6pt)-- ++(-2.5pt,0pt);
}

\newtcolorbox{Box2}[2][]{
                lower separated=false,
                colback=white,
colframe=white!20!black,fonttitle=\bfseries,
colbacktitle=white!30!gray,
coltitle=black,
enhanced,
attach boxed title to top left={yshift=-0.1in,xshift=0.15in},
title=#2,#1}


\begin{document}
\begin{center}
   \shadowbox {\bf{Comportement global d’un circuit }}
\end{center}


%%_________________________Exercice ! :"_________________________Exercice
   \begin{Box2}{Exercice 1 : l’énergie électrique et rondement }

      L’équation de la caractéristique traduisant la loi d’ohm aux bornes d’un générateur est :$ U_{PN} = 1.5 - 2I$

      1. Déterminer la f.é.m. E et la résistance interne r de ce générateur.

      2. On effectue ensuite une étude énergétique dans le cas où le générateur fonctionne durant 10 minutes. La tension à ses bornes est 1V.

      2.1. Calculer l’énergie dissipée par effet joule.

      2.2. Calculer l’énergie fournie par le générateur au reste du circuit.

      2.3. Calculée l’énergie électrique générée.

      2.4. Calculer le rondement de ce générateur, conclure.
   \end{Box2}


%%_________________________Exercice !2 :"_________________________Exercice
\begin{Box2}{Exercice 2 :le point de fonctionnement du circuit. }
Un circuit électrique comporte un générateur (E = 6V; r = 2$\Omega$ )et un électrolyseur
(E' = 2,4V ; r' = 10 $\Omega$ ).

1. Déterminer le point de fonctionnement du circuit.

2. Calculer la puissance électrique engendrée par le générateur, la puissance que peut
recevoir l’électrolyseur et la puissance utile transformée en réactions chimiques.

3. Calculer le rendement du générateur et aussi de l’électrolyseur. Calculer le rendement
du circuit.
\end{Box2}

%%_________________________Exercice ! 3:"_________________________Exercice
\begin{Box2}{Exercice 3 :  Le moteur (M) électrique d’un treuil }

   Le moteur (M) électrique d’un treuil est alimenté par une batterie d’accumulateurs. Cette dernière est considérée comme un générateur de f.é.m. 144 V et de résistance interne 0,1$\Omega$.

1.a. Calculer l’énergie électrique transférée par la batterie au moteur M du treuil si ce dernier est traversé par un courant d’intensité 35 A durant 3 s.

1.b. En déduire le rondement de la batterie.

2. Le treuil soulève, à vitesse constante, un bloc de béton de 630 kg, d’une hauteur de 1,7 m en 3s . Sachant que l’intensité du courant électrique qui traverse le moteur est 35A.

2.a  Calculer la valeur de l’énergie convertie par le moteur en énergie mécanique.
On donne : $g = 9,8 N/kg$

2.b. Quel est le rondement du moteur ?

2.c. En déduire La f.é.m. E’ du moteur M.

3. La résistance interne du moteur est r = 0,4 $\Omega$.

3.a. Calculer l’énergie dissipée par effet joule.
3.b. Le principe de conservation de l’énergie est-il vérifié au niveau du moteur ? Interpréter ce
résultat.
\end{Box2}


\begin{Box2}{Exercice 4 : Schéma énergétique}
   Un générateur de f.é.m. E = 33V débite un courant d’intensité I = 11A lorsqu’il est connecté à un conducteur
ohmique de résistance R = 2,5$\Omega$. Calculer :

   1. la puissance dissipée par effet Joule dans le conducteur ohmique,

   2. la puissance totale disponible dans le générateur,

   3. la puissance dissipée par effet Joule dans le générateur,

   4. la résistance interne du générateur.

   5. Faire un schéma énergétique montrant les transferts d’énergie s’effectuant au niveau de chaque dipôle de
circuit .
\end{Box2}
\begin{center}
   \Large{ \em{Exercices Supplémentaires}}
\end{center}


%%_________________________Exercice 4 : _________________________Exercice
\begin{Box2}{Exercice 5 : les paramètres d’une pile }

On considère le montage dans la figure si dessous :
   Lorsqu’on ferme l’intérrupteur , il passe dans le conducteur ohmique $R_2$ un courant $I_2=0,8A$ et la tension devient $U_{AB}=8V$.($E'_2 = 4V$ , $r'_2$ =$ 3\Omega$) et $r'_1 = 2\Omega$ , $R_1 = 10\Omega$
  \begin{center}
     \vspace{-0.5cm}

  \begin{circuitikz}[european, voltage shift=0.5]
 \draw (0,0) to[vsourceC, name=$E$ , v=E.r] (0,3)
 to[short, -*, f=$I_0$] (1,3)
 to [Telmech=M,n=motor] ++(1,0)
 to[short,-*]  (2.5,3)
     to[PZ,l=$E'_2 \; r'_2$] (3.5,3)
     
 to[short,-*]  (4,3)
     to[R=$R_1$, f>_=$i_1$] (4,0) -- (0,0);
 \draw (4,3) -- (6,3)
 to[R=$R_2$, f>_=$i_2$]
 (6,0) to[short, -*] (2,0); 
     \draw (4,3) node[label=A]{};
     \draw (4,-0.7) node[label=B]{};
  \end{circuitikz}
  \end{center}


   1. Déterminer la valeur de la résistance $R_2$ puis en déduire la valeur de la réstance $R_{eq1/2}$ équivalente aux deux réstances R1et R2
montées en parallèle .

   2. Déterminer l’intensité $I_1$ puis en déduire la valeur de l’intensité $I$ du curant qui passe dans le circuit. 

   3. Déterminer la valeur de la résistance interne du générateur sachant que la puissance thermique dissipée dans tout le circuit
est : $P_J=38,4W$

   4. Caculer la puissance utile $P_{u2}$ dans l’éléctrolyseur.

   5. Déduire la valeur de puissance totale $P_t$ du générateur sachant que la puissance utile du moteur est $P_{u1}=6W$

   6. Déterminer la valeur de la force électromotrice E du générateur

   7. Déterminer la valeur de la force contre électromotrice E’1 de l’électrolyseur

   8.Caculer le rendement du générateur
\end{Box2}
%\vspace{2cm}
%%_________________________Exercice 5 : _________________________Exercice
\begin{Box2}{Exercice 6 : bilan énergétique}
  
   \begin{wrapfigure}[6]{r}{0.5\textwidth}

      \vspace{-1cm}
      \begin{center}
   \begin{circuitikz}[european, voltage shift=0.5]
 


  
    \draw (0,0)
    to[american voltage source,  i=$I_0$, v=$U_0$] (0,3)
    to[short] (1,3)
    to[R=$R_{Shunt}$, i=$I_0$, v=$U_S$] (3,3)
    to[short, -o] (4,3);
    
    \draw (0,0) to[short, -o] (4,0);
    \draw [dashed, gray] (4,3) to [Telmech=M,n=motor] ++(0,-3);
  \end{circuitikz}
  \end{center}


\end{wrapfigure}


   On considère le montage suivant constitué :
\\-d’Un générateur de force électromotrice E et de résistance interne r et un intérrupteur .
\\-d’un moteur de force électromotrice E’=2,4V et de résistance interne $r'=2\Omega$ et d’un fil inextensible enroulé sur la poulie du moteur et auquel est suspendu à l’autre extrémité un corps de masse m=50g.
\\-d’un conducteur ohmique de résistance $R=30\Omega$.

On ferme l’intérrupteur et il passe dans le circuit un courant électrique d’intensité I=0,1A.
  
   1. Déterminer la puissane $P_J$ déssipée par effet joule dans l’ensemble : ( le conduceur ohmique + le moteur).

2. Calculer la puissance utile du moteur électrique.

3 En déduire la puissance Pe fournie par le générateur au reste du circuit.

4 Sachant que la puissance totale déssipée dans tout le circuit par effet joule est égale à  0,36W .

4.1. Déteminer la valeur de la puissance déssipée par effet joule dans le moteur.

4.2. En déduire la valeur de la résistance du générateur.

5. Déterminer la valeur de la force électromotrice du générateur puis retrouver l’intensité du courant en utilisant la loi de
pouillet.

6. Sachant que l’énergie utile reçue par le moteur se transforme en énergie potentielle de pesanteur ce qui entraine la montée du
corps S d’une distance h pendant une durée $\Delta{t} = 2s.$

7. Déterminer la valeur de h .on donne g=10N/kg.

   8. Quelles sont les formes d’énergie qui ont été mis en évidence dans cette expérience ?
\end{Box2}
%


%%_________________________Exercice 6 : _________________________Exercice
\end{document}
