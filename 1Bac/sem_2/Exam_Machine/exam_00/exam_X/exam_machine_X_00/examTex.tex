\documentclass[12pt]{article}
\usepackage[a4paper, total={7.5in, 10in}]{geometry}
%\usepackage{array}
\usepackage{graphicx, subfig, wrapfig, fancyhdr, lastpage, multicol ,color}
\newcommand\headerMe[2]{\noindent{}#1\hfill#2}
\usepackage[mathscr]{euscript}

\setlength{\columnseprule}{1pt}
\def\columnseprulecolor{\color{blue}}


\pagestyle{fancy}
\fancyhf{}

\cfoot{\em{Page \thepage \hspace{1pt} / \pageref{LastPage}}}
\begin{document}

\headerMe{Royaume du Maroc}{année scolaire \emph{2021-2022}}\\
\headerMe{Ministère de l'Éducation nationale, }{  Professeur :\emph{Zakaria Haouzan}}\\
\headerMe{du Préscolaire et des Sports}{Établissement : \emph{Lycée SKHOR qualifiant}}\\

\begin{center}
Devoir Surveillé  N°1 \\
    Filière 1Bac Sciences Expérimentales\\
Durée 2h00
\\
    \vspace{.2cm}
\hrulefill
\Large{Chimie 7pts/42min}
\hrulefill\\

    %\emph{Les questions parties sont indépendantes}
\end{center}
%end Headerss------------------------
%__________________Chimie ______________________-
%%%%%%%+_+_+_+_+_+_+_+_+_Partie1

 \section*{Partie 1 :Suivi d’une transformation chimique \dotfill(3pts) }
   Une solution aqueuse d’acide chlorhydrique $H^+_{(aq)} + Cl^{-}_{(aq)}$ réagit avec le magnésium solide $Mg(s)$ . on obtient un dégagement de dihydrogène et il se forme des ions magnésium $Mg^{2+}_{(aq)}$.
   \begin{enumerate}
   \item  Écrire l’équation de la réaction \dotfill(1pt)
   \item On introduit dans un flacon une masse m = 27g de magnésium et on ajoute 40ml de solution d’acide chlorhydrique de concentration molaire C = 1, 0 mol/l , on bouche rapidement le flacon et en utilisant un manomètre digitale, on mesure la pression finale dans le flacon .
      \begin{enumerate}
        \item Déterminer les quantités initiale des réactifs \dotfill(1pt)
        \item À l’aide d’un tableau d’avancement , déterminer l’avancement final et le réactif limitant .\dotfill(1pt)
      \end{enumerate}
   \end{enumerate}

 \section*{Partie 2 :la Mesure de la conductance G et la conductivité \dotfill(4pts) }
On prépare une solution aqueuse en dissolvant une masse $m=271mg$ de chlorure de fer III $FeCl_3$ anhydre dans un volume $V=250mL$ d’eau distillée .

\begin{enumerate}
    \item Ecrire l’équation de dissolution de $FeCl_3$ dans l’eau.\dotfill(0.5pt)
    \item Déterminer la concentration c de la solution obtenue en $mol/L$ puis en $mol/m^3$ . On  donne :  $ M( FeCl_3 )=162,5g/mol$ \dotfill(0.5pt)
    \item Déterminer les concentrations effectives des espèces chimiques qui se trouvent dans la solution .\dotfill{0.5pt}
\end{enumerate}
Pour mesurer la conductance de cette solution ,on utilise une cellule conductimétrique qui se compose d’un générateur GBF ,
de deux plaques conductrices en regard séparées d’une distance L=2cm , la surface de chacune d’elles est  $S=4cm^2$ qui sont complétement immergées dans la solution , d’un ampéremètre pour mesurer l’intensité du courant dans le circuit et d’un voltmètre monté entre les borne des plaques .

\begin{enumerate}

  \item Faites un schéma du montage utilisé dans la cellule conductimétrique \dotfill(0.5pt)
  \item Sachant que l’intensité du courant électrique dans le circuit est $I=83mA$ et la tension entre les plaques de la cellule est :$U=25V$ , déterminer la valeur de la conductance de la cellule.\dotfill(0.5pt)
  \item En déduire la valeur de la conductivité $\sigma$ de la solution \dotfill(0.5pt)
  \item   Donner l’expression de la conductivité de la solution en fonction de la concentration c et de la conductivité molaire
    ionique des espèces ioniques présents dans la solution .\dotfill(1pt)
\end{enumerate}
 %_____________________________________PHYSIque Partie 22222____________________________________________________________________________
\begin{center}
    \vspace{2cm}
\hrulefill
\Large{Physique 13pts - 78min}
\hrulefill\\
    \emph{Les parties sont indépendantes}
\end{center}
%end Headerss------------------------


\begin{multicols}{2}
    [
        \section*{Questions du cours: Choisir la bonne réponse. (5 pts)}
    ]

\begin{enumerate}
  \item Puissance électrique reçue par un récepteur :  \dotfill(1pt)
      \begin{enumerate}
          \item est égale au produit de la
            tension $U_{AB}$ à ses bornes par l’intensité du courant qui le traverse
          \item $\mathscr{P_e} = \frac{W}{\Delta{t}}$
          \item $\mathscr{P_e} = \frac{U.I.\Delta{t}}{\Delta{t}}$
      \end{enumerate}

%  \item  Unité de la puissance d’une force est :
      %\begin{enumerate}
          %\item[] a)Joule \hspace{1cm}  b)Newton  \hspace{1cm}c)Watt
      %\end{enumerate}

    \item La puissance électrique reçue par un récepteur : \dotfill(1pt)
      \begin{enumerate}
          \item dépend de l’intensité dans le circuit.
          \item dépend du temps de fonctionnement.
          \item dépend de la tension dans le circuit.
          \item est exprimée en Watt (W).
      \end{enumerate}
        \vspace{1cm}
      \item La tension entre les bornes d’un générateur est de 20V. Il est parcouru par un courant d’intensité 10A. Le temps de fonctionnement est de 10 minutes. La puissance électrique fournie est :\dotfill(1pt) 
      \begin{enumerate}
          \vspace{-0.6cm}
        \item[] a)200J \hspace{0.25cm} b)200W \hspace{0.25cm} c)2000J \hspace{0.25cm} d)0,2kW
      \end{enumerate}

    \item  L'effet joule :\dotfill(2pt)
      \begin{enumerate}
          \item  est une conversion d'énergie thermique en énergie électrique.
          \item  est une conversion d'énergie électrique en énergie thermique.
          \item  est associé au passage du courant électrique dans un conducteur.
       \end{enumerate}

\end{enumerate}
\end{multicols}

 \section*{Partie 2 :  Transfert de l’énergie dans un circuit électrique- Puissance électrique \dotfill(8pts)}
Une batterie d’accumulateur au plomb est chargée de $40 Ah$.
   \begin{enumerate}
   \item La batterie se décharge complètement en $1 h$. La tension au cours de cette décharge est $11,8 V$. Quelle est
     l’énergie électrique fournie ?\dotfill(1pt)
\item On utilise la batterie pour démarrer une automobile pendant $1,5 s$. La batterie est alors traversée par un
  courant d’intensité $0,2 kA$ et la tension à ses bornes est de $10,2 V$.
         \begin{enumerate}
           \item Quelle est l’énergie électrique fournie ?\dotfill(1pt)
           \item Quelle est la puissance électrique ?\dotfill(1pt)
         \end{enumerate}
       \item Une génératrice de courant continu convertit une puissance mécanique de $P_m= 1,86 kW$ en énergie électrique.
La tension à ses bornes est de $112V$ et elle débite un courant d’intensité $14,2 A$.
   \begin{enumerate}
     \item Calculer la puissance électrique fournie par cette génératrice.\dotfill(1pt)
     \item Calculer la puissance dissipée par effet Joule.\dotfill(1pt)
   \end{enumerate}
 \item Une batterie d’accumulateur au plomb alimente les lampes d’une automobile. La tension entre les bornes de la
batterie est de $11,9 V$ et l’intensité du courant qui passe dans la batterie est $10,3 A$.
   \begin{enumerate}

     \item Quelle est la puissance électrique fournie par la batterie ?\dotfill(1pt)

\item Dans ces conditions, le fonctionnement de la batterie dure 17 min. Quelle est l’énergie électrique transférée
  dans les circuits récepteurs ? \dotfill(2pt)
   \end{enumerate}
   \end{enumerate}
\end{document}
