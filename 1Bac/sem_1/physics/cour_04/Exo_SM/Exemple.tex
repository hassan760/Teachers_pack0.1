\documentclass[12pt, french]{article}

\usepackage{fancyhdr, fancybox, lastpage}
\usepackage[most]{tcolorbox}
\usepackage[a4paper, margin={0.3in, .75in}]{geometry}
\usepackage{wrapfig}
\pagestyle{fancy}
\renewcommand\headrulewidth{1pt}
\renewcommand\footrulewidth{1pt}
\fancyhf{}
\rhead{ \em{Zakaria Haouzan}}
\lhead[C]{\em{1ére année baccalauréat Sciences Mathématiques}}
\chead[C]{}
\rfoot[C]{}
\lfoot[R]{}
\cfoot[]{\em{Page \thepage / \pageref{LastPage}}}


\newtcolorbox{Box2}[2][]{
                lower separated=false,
                colback=white,
colframe=white!20!black,fonttitle=\bfseries,
colbacktitle=white!30!gray,
coltitle=black,
enhanced,
attach boxed title to top left={yshift=-0.1in,xshift=0.15in},
title=#2,#1}


\begin{document}
\begin{center}
   \shadowbox {\bf{Travail et énergie interne (Sc. Math) }}
\end{center}


%%_________________________Exercice ! :"_________________________Exercice
   \begin{Box2}{Exercice 1 : }
   1. Définir :  l'énergie interne du système  et la La variation de l'énergie interne du système
     \\2. Si l’état final est identique à l'état initial lors d’une transformation, on dit que le système a subit une transformation cyclique.Quelle est la valeur de la variation de l’énergie interne dans ce cas ?justifier votre réponse.
\\3. Au cours d’une transformation, un gaz reçoit une énergie de 100J du milieu extérieur. Quel est l’effet de ce transfert d’énergie sur le gaz?
\\4. Le gaz revient à son état initial sans transfert d’énergie par travail. Dans cette étape, sous quelle forme y'
a-t-il transfert d’énergie entre le gaz et le milieu extérieur ? Expliquer qualitativement ce transfert. Y'a t-il
un avantage pratique à ce transfert ?
\\5. Un système reçoit au cours d’une transformation 50J par travail et cède 70J par transfert thermique. Trouver la
variation de l’énergie interne du système. 

   \end{Box2}


%%_________________________Exercice !2 :"_________________________Exercice
\begin{Box2}{Exercice 2 : }

On considère un système adiabatique (qui n’échange pas la chaleur avec le milieu extérieur) formé de  (cylindre
+ piston),le piston a un rayon r=2cm.A l’intérieur du cylindre se trouve un gaz parfait son volume V0 et sa
température T0 et la pression du gaz P0=Patm= 105
Pa.
On applique sur piston une force F constante son intensité F=19N, il descend lentement à vitesse constante sans
frottement et parcourt la distance d=1cm jusqu'à ce que la pression devient P1 et le volume devient V1 à la même
température T0.
  \begin{enumerate}
    \item Définir l’énergie interne d’un système.

\item Donner l'énoncé du premier principe de la thermodynamique.

  \item Calculer la pression du gaz P1 à l’état final.

  \item Trouver l’expression de la force qu’applique le milieu extérieur sur le piston en fonction de V0, V1et P1.

  \item Calculer la variation de l’énergie interne du gaz au cours de cette transformation.

  \item Trouver les valeurs de V0 et V1.

  \item On élimine la force F et on met au-dessus du piston une masse m de façon que le gaz garde le même
volume V1 et sa pression est P1=Pg et sa température est T0.quelle est la valeur de m ? on donne
g=10N/kg.
  \end{enumerate}
\end{Box2}

%%_________________________Exercice ! 3:"_________________________Exercice
\begin{Box2}{Exercice 3 :}
Fatima veut prendre un bain à 35◦C. Elle fait couler 100l d’eau chaude à 65◦C, provenant de
son cumulus électrique. Trouvant alors son bain trop chaud, elle y ajoute de l’eau froide à 20◦C.
  \begin{enumerate}
      \item Quel est le mode de transfert d’énergie de l’eau chaude vers l’eau froide ?
      \item Si les pertes énergétiques sont négligeables, quel volume d’eau froide faut-il ajouter ?
\item  Quels autres échanges énergétiques faudrait-il considérer en réalité ? Le volume d’eau

froide réel à ajouter est-il plus ou moins grand que le résultat trouvé à la question précédente ?
  \end{enumerate}
\end{Box2}

\vspace{2cm}
\begin{center}
   \Large{ \em{Exercices Supplémentaires}}
\end{center}
%%_________________________Exercice 4 : _________________________Exercice
\begin{Box2}{Exercice 4 : }
Choisir la proposition vraie :
On fournit 50J à un système, par travail et le système cède au milieu extérieur 100J sous forme
d’énergie thermique.
  \begin{enumerate}
    \item L’énergie reçue par le système est :
      \begin{enumerate}
        \item W = -50J
        \item W = 50J
      \end{enumerate}
    \item l’énergie cédée par le système au milieu extérieur est :
       \begin{enumerate}
        \item Q = -100J
        \item Q = 100J
      \end{enumerate}
    \item la variation de l’énergie interne est :
 \begin{enumerate}
        \item $\Delta$U = -150J
        \item $\Delta$U = -50J
        \item $\Delta$U = 50J
      \end{enumerate}
      \item On considère un système énergétiquement isolé et siège des frottements
        \begin{enumerate}
          \item l’énergie mécanique de ce système est constante
          \item l’énergie interne de ce système ne varie pas
            \item la variation de l’énergie mécanique du système est égale à l’opposée de la variation de son
énergie interne
\item la variation de l’énergie cinétique du système est égale à l’opposée de la variation de son
énergie potentielle de pesanteur
\item le système s’échauffe.
        \end{enumerate}
        \item Répondre vrai ou faux en justifiant votre réponse :
          \begin{enumerate}
            \item Le travail d’une force ne peut que faire varier l’énergie cinétique d’un système.
              \item Il est possible d’élever la température d’un corps sans chauffage.
                \item Les particules constituant un solide cristallin sont immobiles dans un réseau cristallin.
                \item Vaporiser un liquide augmente le désordre des molécules qui le constituant.
                \item L’énergie stockée dans un système est l’énergie interne.
                  \item Dans le système international des unités, l’unité de l’énergie transférée, par le travail, à un
système est le joule (J).
\item L’énergie transférée par le travail, à un système peut faire augmenter la température du
système.
          \end{enumerate}
  \end{enumerate}
\end{Box2}




%%_________________________Exercice 6 : _________________________Exercice
\end{document}
