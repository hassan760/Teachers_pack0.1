\documentclass[12pt, french]{article}

\usepackage{fancyhdr, fancybox, lastpage}
\usepackage[most]{tcolorbox}
\usepackage[a4paper, margin={0.3in, .75in}]{geometry}
\usepackage{multirow}

\pagestyle{fancy}
\renewcommand\headrulewidth{1pt}
\renewcommand\footrulewidth{1pt}
\fancyhf{}
\rhead{ \em{Zakaria Haouzan}}
\lhead[C]{\em{1ére année baccalauréat Sciences Expérimentales}}
\chead[C]{}
\rfoot[C]{}
\lfoot[R]{}
\cfoot[]{\em{Page \thepage / \pageref{LastPage}}}


\newtcolorbox{Box2}[2][]{
                lower separated=false,
                colback=white,
colframe=white!20!black,fonttitle=\bfseries,
colbacktitle=white!30!gray,
coltitle=black,
enhanced,
attach boxed title to top left={yshift=-0.1in,xshift=0.15in},
title=#2,#1}


\begin{document}
\begin{center}
   \shadowbox {\bf{Les Réactions Acido-basiques}}
\end{center}


%%_________________________Exercice ! :"_________________________Exercice
   \begin{Box2}{Exercice 1 :Solution d’acide Lactique } 

  On mélange un volume $V_1$ = 12,0 mL d’une solution d’acide lactique $CH_3CH(OH)CO_2H$, noté AH, de
concentration $C_1$ = 0,16 mol/L avec un volume $V_2$ = 23,0 mL d’une solution basique de méthylamine ${CH_3NH_2}_{(aq)}$ de concentration $C_2 = 5.10^{-3} mol/L$ .

1- Ecrire les couples acide/base et les demi-réactions acido-basiques relatives.
\\2- Ecrire l’équation de la réaction qui peut se produire.
\\3- Etablir la composition finale du système en quantité de matière, puis en concentrations.
   \end{Box2}

%%_________________________Exercice !2 :"_________________________Exercice
\begin{Box2}{Exercice 2 : L’acide benzoïque et le benzoate de sodium }

   L’acide benzoïque $C_6H_5COOH$ et le benzoate de sodium $C_6H_5COONa$ sont utilisés comme conservateurs, notamment dans les boissons dites  "light". Ils portent les codes respectifs E210 et
E211.

1- Ecrire l’équation de dissolution du benzoate de sodium dans l’eau.

2- Identifier le couple acide/base mettant en jeu l’acide benzoïque et écrire la demi-équation acido-
basique correspondante.

   3- On fait réagir une masse m = 3,00 g d’acide benzoïque avec 150 mL d’une solution d’hydroxyde de sodium de concentration $C = 2,50.10^{-1}mol.L^{-1}.$

3.1- Identifier les couples acide/base mise en jeu, puis écrire l’équation de la réaction envisagée.

3.2- Etablir un tableau d’avancement et déterminer maximal de la réaction. 
  \\ Quel est le réactif limitant ?
\end{Box2}

%%_________________________Exercice 3 : _________________________Exercice
\begin{Box2}{Exercice 3 : Vitamine C }
Les comprimés effervescents de vitamine C contiennent de l’acide ascorbique $C_6H_8O_6$ (E300) et l’ascorbate de sodium $NaC_6H_7O_6$ (E301) est le sel de sodium de la vitamine C , ce dernier est
employé comme additif alimentaire.

   1- Écrire l’équation de dissolution d’ascorbate de sodium dans l’eau.

   2- Identifier le couple acide / base mettant en jeu l’acide ascorbique et écrire la demi-équation
acido-basique correspondante.

   3- On fait réagir une masse m = 3,00 g d’acide ascorbique avec 150 mL d’une solution d’hydroxyde de sodium ($Na^+$ , $HO^-$)de concentration $C=2,50.10^{-1}mol.L^{-1}$.

a) Identifier les couples acide / base mis en jeu, puis écrire l’équation de la réaction envisagée.

   b) Établir un tableau d’avancement et déterminer l’avancement maximal de la réaction. Quel est
le réactif limitant ?
\end{Box2}

%%_________________________Exercice 4 : _________________________Exercice
\begin{Box2}{Exercice 4 : Vitamine B5}
   Les comprimés effervescents de Vitamine B5, contiennent acide pantothénique $C_9H_{17}NO_5$ et le
   pantothénate de sodium $NaC_9H_{16}NO_5$ est le sel de sodium de la vitamine B5 , ce dernier est
employé comme additif alimentaire.

   1- Écrire l’équation de dissolution de pantothénate de sodium dans l’eau.

2- Identifier le couple acide / base mettant en jeu l’acide pantothénique et écrire la demi-équation acido-basique correspondante.

3- On fait réagir une masse m = 3,00 g d’acide pantothénique avec 150 mL d’une solution
d’hydroxyde de sodium $(Na^+, HO^-)$de concentration $C=2,50.10^{-1} mol.L^{-1}$.

a) Identifier les couples acide / base mis en jeu, puis écrire l’équation de la réaction envisagée.

b) Établir un tableau d’avancement et déterminer l’avancement maximal de la réaction. Quel est
le réactif limitant ?
\end{Box2}

\vspace{3cm}
\begin{center}
   \Large{ \em{Exercices Supplémentaires}}
\end{center}
%%_________________________Exercice 6 : _________________________Exercice

%%_________________________Exercice 7 : _________________________Exercice

\begin{Box2}{Exercice 5 : Relation entre les conductivités ioniques de différentes solutions }
   On mélange un volume $V_1 = 25,0 mL$ d’une solution d’acide acétique ${CH_3CO_2H}_{(aq)}$ à $C_1$=$2,50.10^{-2} mol.L^{-1}$
et un volume $V_2 = 75,0 mL$ d’une solution de borate de sodium ${Na^+}_{ (aq)} + {BO_2}_{(aq)}$ à $C_2 = 1,00.10^{-2} mol.L^{-1}$.

1- L’ion borate est une base. Ecrire la demi-équation acido-basique correspondante.

2- Calculer les quantités initiales d’acide éthanoïque et d’ions borate présents dans le mélange.
   La réaction qui se produit lors du mélange a pour équation :$$ {CH_3CO_2H}_{(aq)} + {BO_2^-}_{(aq)} \rightarrow {CH_3CO_2^-}_(aq)+ {HBO_2}_{(aq)}$$

3) A l’aide d’un tableau d’avancement, déterminer la composition finale en quantités, puis en
concentration du mélange.
\end{Box2}


\begin{Box2}{Exercice 6 : l’entretien des eaux de piscine}
Une poudre utilisée pour l’entretien des eaux de piscine contient, de l’hydrogénosulfate de sodium de formule $NaHSO_4$. Donnée : $M(NaHSO_4) = 120 g.mol^{-1}$

1- L’ion hydrogénosulfate, présent dans la poudre, se comporte comme un acide. Écrivez le couple acide/base
auquel il appartient et sa demi-équation de couple.

   2- Vous dissolvez $2,50g$ de cette poudre dans $V=100mL$ d’eau. Écrivez l’équation de dissolution de
l’hydrogénosulfate de sodium.

   3- Vous faites réagir les ions hydrogénosulfate de la solution obtenue avec des ions hydroxyde. Les conditions
de la transformation chimique sont stoechiométriques lorsque vous avez versé $V_b = 18,0mL$ d’une solution
d’hydroxyde de sodium de concentration $C_b =1,00.10^{-1} mol.L^{-1}$.

a) Écrivez le deuxième couple acide-base intervenant dans cette réaction et sa demi-équation de couple.

   b) Ecrivez les deux demi-équations de réaction et l’équation-bilan de la réaction.

   c) Exprimez et calculez l’avancement maximal de la réaction.

   d) Exprimez et calculez la concentration en ions sulfate.

   e) Calculez la masse d’ hydrogénosulfate de sodium qui était présente dans les 2,50 g de poudre.
\end{Box2}

%%______________

%%_________________________Exercice ! 3:"_________________________Exercice
\begin{Box2}{Exercice 3 : Hydrogénocarbonate de sodium}
   On introduit une masse $m=0,50g$ d'hydrogénocarbonate de sodium, de formule $NaHCO_3$, dans un erlenmeyer et on ajoute progressivement de l'acide chlorhydrique $({H_3O^+}_{(aq)} + {Cl^-}_{(aq)})$ (solution aqueuse de chlorure d'hydrogène).

   1- Ecrire l’équation de dissolution d'hydrogénocarbonate de sodium dans l’eau.

   2- Les coulpes acides base mise en jeu ,sont :$({H_3O^+}_{(aq)}/{H_2O}_l $ et $(CO_2,H_2O)/{HCO_3^-}_{(aq)})$

   3-Donner la demi-équation acido-basique relative à chaque couple.

   4-Déduire l'équation de la réaction qui se produit dans l'erlenmeyer.

   5- Donner le nom du gaz qui se dégage au cours de la transformation (dioxyde de carbone /
dihydrogène)

   6- Dresser le tableau d’avancement

   7- Quel volume V d'acide chlorhydrique de concentration $C=0,10mol.L^{-1}$ faut-il verser pour que le
dégagement de gaz cesse ?

   8- Quel est alors le volume de gaz dégagé si le volume molaire dans les conditions de l'expérience est
   $V_m=24,0 L.mol^{-1}$ ?

   Données : masses molaires $M(Na) = 23 g.mol^{-1}$ , $M(C) = 12 g.mol^{-1}$ , $M(O) = 16 g.mol^{-1}$ , $M(H)$=$1 g.mol^{-1}$



\end{Box2}






\end{document}
