\documentclass[12pt, french]{article}

\usepackage{fancyhdr, fancybox, lastpage}
\usepackage[most]{tcolorbox}
\usepackage[a4paper, margin={0.3in, .75in}]{geometry}
\usepackage{multirow}

\pagestyle{fancy}
\renewcommand\headrulewidth{1pt}
\renewcommand\footrulewidth{1pt}
\fancyhf{}
\rhead{ \em{Zakaria Haouzan}}
\lhead[C]{\em{1ére année baccalauréat Sciences Mathématiques}}
\chead[C]{}
\rfoot[C]{}
\lfoot[R]{}
\cfoot[]{\em{Page \thepage / \pageref{LastPage}}}


\newtcolorbox{Box2}[2][]{
                lower separated=false,
                colback=white,
colframe=white!20!black,fonttitle=\bfseries,
colbacktitle=white!30!gray,
coltitle=black,
enhanced,
attach boxed title to top left={yshift=-0.1in,xshift=0.15in},
title=#2,#1}


\begin{document}
\begin{center}
   \shadowbox {\bf{Mesure de la conductance}}
\end{center}


%%_________________________Exercice ! :"_________________________Exercice
   \begin{Box2}{Exercice 1 : Caractéristiques géométriques de la cellule conductimètrique.}
  
On vérifie la constante k=S/l (S : surface de la cellule et l : écartement des deux électrodes) de la
      cellule d’un conductimètre en la plongeant dans une solution étalon de chlorure de potassium à $10^{ -1}$ mol/L. A la température de l’expérience, la conductivité de cette solution est égale à $1,191 S.m^{-1}$. 

Le conductimètre indique une conductance de $14,3.10^{-3}$ S.
\\1. Quelle est la valeur de la constante ?
\\2. Si les deux électrodes, planes et parallèles, sont séparées de 1 cm, quelle est leur surface ?

   \end{Box2}


%%_________________________Exercice !2 :"_________________________Exercice
\begin{Box2}{Exercice 2 : Mesure d’une conductance }
   Aux bornes d’une cellule plongée dans une solution de chlorure de potassium et branchée sur un
générateur alternatif, on a mesuré une tension efficace de 13,7 V et une intensité efficace de 89,3mA.

1. Calculer la résistance R de la portion d’électrolyte comprise entre les électrodes.

2. Calculer la conductance G en S.

3. La conductivité de cette solution est de 0,512 mS/cm à 20°C. Calculer la valeur de la constante k de
cellule définie par : $G=K\sigma$

\end{Box2}

%%_________________________Exercice ! 3:"_________________________Exercice
\begin{Box2}{Exercice 3 : Conductivité et conductance }

 On mélange 200 mL de solution de chlorure de potassium à $5,0.10^{-3} mol.L^{-1}$ et 800 mL de solution de chlorure de sodium à $1,25.10^{-3} mol.L^{-1}$. 

   1.Quelle est la conductivité de la solution obtenue ?
   
   2. Dans le mélange précédent, on place la cellule d’un conductimètre. La surface des électrodes est de
$1,0 cm^2$ et la distance les séparant de 1,1 cm. Quelle est la valeur de la conductance G ?

\end{Box2}

%%_________________________Exercice 4 : _________________________Exercice
\begin{Box2}{Exercice 4 : Conductance et conductivité molaire ionique}

   La conductance d’une solution de chlorure de sodium, de concentration $C_1=0,150 mol.L^{-1}$,est $G_1$=$2,188.10^{-2}S$. On mesure la conductance $G_2$ d’une deuxième solution de chlorure de sodium avec le même conductimètre. On obtient $G_2= 2,947.10^{-2} S$. 
   
   1. Calculer la concentration molaire $C_2$ de cette deuxième solution. La température du laboratoire et des solutions est de 25°C.

   2. La constante de la cellule du conductimètre est $k = 86,7 m_{-1}$. La distance entre les électrodes de la cellule est L = 12,0 mm. Calculer l’aire S de chaque électrode.

   3. Calculer la conductivité $\sigma$ de la première solution.

   4.La conductivité molaire ionique de l’ion sodium $Na^+$ est $\lambda_{Na^+} = 50,1.10^{-4} S.m^2.mol^{-1}.$ Déterminer la conductivité molaire ionique $\lambda_{Cl^-}$ de l’ion chlorure $Cl^-$
\end{Box2}

\vspace{3cm}
\begin{center}
   \Large{ \em{Exercices Supplémentaires}}
\end{center}
%%_________________________Exercice 5 : _________________________Exercice
\begin{Box2}{Exercice 5 : Dosage par étalonnage d'une solution de chlorure de potassium }
L’hypokaliémie désigne une carence de l’organisme en élément potassium ,  pour compenser rapidement cette 
   carence, on peut utiliser une solution de chlorure de potassium, injectable par voie intraveineuse. 
   
le chlorure de potassium Lavoisier, par exemple, est proposé en ampoules de 20 mL contenant une  masse m  de KCl. Pour déterminer cette masse m, on dispose d’une solution étalon de chlorure de potassium Se à 
   $10 mmol.L^{-1}$ et d’un montage conductimètrique.

\begin{center}
\begin{tabular}{ |l|c|c|c|c|c|c| }
\hline
   C (mmol/L)  & 1,0  & 2,0  & 4,0  & 6,0  & 8,0  & 10,0 \\ \hline

   {G( mS) }   & 0,28 & 0,56 & 1,16 & 1,70 & 2,28 & 2,78 \\

\hline
\end{tabular}
\end{center}

1. Pour étalonner la cellule conductimètrique, on prépare à partir de la solution étalon Se, cinq
solutions filles Si de volume V = 50,0 mL et de concentrations respectives 8,0 ; 6,0 ; 4,0 ; 2,0 ; et 1,0
   $mmol.L^{-1}$

   Tracer la courbe G = f ( c ) à l’aide des données du tableau ci-dessus. Conclure.

2. On mesure, avec ce montage et à la même température, la conductance de la solution de l’ampoule.
On obtient : $G_a = 293 mS$. Peut-on déterminer directement la concentration en chlorure de potassium de
l’ampoule injectable grâce à cette courbe ? Justifier la réponse.

3. Le contenu d’une ampoule a été dilué 200 fois. La mesure de sa conductance donne : $G_d$ = 1,89 mS.
En déduire la valeur de la concentration $C_d$ de la solution diluée, puis celle de la solution de l’ampoule.
Calculer la masse m.

\end{Box2}

%%_________________________Exercice 6 : _________________________Exercice

\begin{Box2}{Exercice 6 : Mélange de deux solutions électrolytiques }

   On dispose d’un volume $V_1=100mL$ d’une solution aqueuse $S_1$ de chlorure de potassium et d’un
volume $V_2=50,0mL$ d’une solution aqueuse $S_2$ de chlorure de sodium. La concentration molaire de la
   solution $S_1$ est égale à $C_1=1,5.10^{-3}mol.L^{-1}$ et la concentration molaire de la solution $S_2$ est égale à $C_2=1,3.10^{-3}mol.L^{-1}$ .

   1. Calculer les conductivités $\sigma_1$ et $\sigma_2$ de chacune de ces solutions.

2. On mélange ces deux solutions.Calculer la concentration molaire de chaque ion dans le mélange.

3. Calculer la conductivité $\sigma$ du mélange.

4. Quelle serait la valeur de la conductance mesurée à l’aide d’électrodes de surface $S=1,0cm^2$ , distantes de L=5,0mm ?

Données  à 25°C: $\lambda_{(K^+)}=7,35.10^{-3} mS.m^2.mol^{-1}$; $\lambda_{(Cl^-)}=7,63.10^{-3} mS.m^2.mol^{-1}$ ; $\lambda_{(Na^+)}$=$5,01. 10^{-3}mS.m^2.mol^{-1}.$



\end{Box2}


%%_________________________Exercice 7 : _________________________Exercice

\begin{Box2}{Exercice 7 : Relation entre les conductivités ioniques de différentes solutions }

   La conductivité d’une solution de $(K^+ + Cl^-)$, de concentration C, est de $114,3\mu S.cm^{-1}$, mesurée à la température du laboratoire. On a mesuré, la même température, les conductivités d’autres solutions à la même concentration :  $(Na^+ + Cl^- ), (K^+ + I^- ), (Na^+ + I^- )$. On a trouvé : $96,2 \mu S.cm^{-1}$, $114,9 \mu S.cm^{-1}$, $95,7 \mu S.cm^{-1}$ respectivement.

1. Attribuer à chaque solution sa conductivité. Justifier la réponse.

2. quelle relation a-t-on entre les conductivités des solutions suivantes:$(Na^+ + Cl^- ), (K^+ + I^- ), (Na^+ + I^- )$

   3. La concentration de ces solutions est-elle de $0,8.10^{-3} mol.L^{-1}$ ou de $8.10^{-3} mol.L^{-1}$ ? Justifier la réponse.


   Données  à 25°C: $\lambda_{(K^+)}=7,35.10^{-3} mS.m^2.mol^{-1}$; $\lambda_{(Cl^-)}=7,63.10^{-3} mS.m^2.mol^{-1}$ ; $\lambda_{(Na^+)}$=$5,01. 10^{-3}mS.m^2.mol^{-1}$ ; $\lambda_{(I^-)} = 7,70 mS.m^2.mol^{-1}$


\end{Box2}

%%______________
\end{document}
