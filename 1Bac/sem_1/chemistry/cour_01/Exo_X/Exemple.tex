\documentclass[12pt, french]{article}

\usepackage{fancyhdr, fancybox, lastpage}
\usepackage[most]{tcolorbox}
\usepackage[a4paper, margin={0.3in, .75in}]{geometry}

\pagestyle{fancy}
\renewcommand\headrulewidth{1pt}
\renewcommand\footrulewidth{1pt}
\fancyhf{}
\rhead{\em{Zakaria Haouzan}}
\lhead[C]{\em{1ére Année Baccalauréat Sciences Expérimentales}}
\chead[C]{}
\rfoot[C]{}
\lfoot[R]{}
\cfoot[]{\em{Page \thepage / \pageref{LastPage}}}


\newtcolorbox{Box2}[2][]{
                lower separated=false,
                colback=white,
colframe=white!20!black,fonttitle=\bfseries,
colbacktitle=white!30!gray,
coltitle=black,
enhanced,
attach boxed title to top left={yshift=-0.1in,xshift=0.15in},
title=#2,#1}


\begin{document}
\begin{center}
   \shadowbox {\bf{ Les grandeurs physiques liées à la quantité de matière
}}
\end{center}


%%_________________________Exercice 1 : _________________________Exercice
\begin{Box2}{Exercice 1 : }
1. Calculer les masses molaires moléculaires des molécules suivantes :
   \begin{center}
   $CO_2$ - $NaCl$ - $H_2SO_4$ - $H_2$ - $SO_2$ - $Al_2(SO_4)_3$ - $N_2O_4$ - $Na_2SO_4$ - $Pb[NO_3]_2$
   \end{center}
2. Déterminer la quantité de matière contenue dans un échantillon de fer $(Fe)$ de masse $11,2 g.$
\\3. Déterminer la quantité de matière que renferme $11,2 L$ de gaz $CO_2$.
\\4. Déterminer la quantité de matière contenue dans $0,1 kg$ de chlorure de sodium ($NaCl$).
\\5. Déterminer la quantité de matière contenue dans un échantillon de nitrate de plomb ($Pb(NO_3)_2$) de masse $9,93 g$.
\\6. Déterminer la masse de $0,6\; mole$ d’acide sulfurique $(H_2SO_4)$.
\\7. Déterminer le volume de $3,2\; moles$ de gaz dihydrogène $(H_2)$.
\\8. Déterminer le volume molaire du mercure sachant que $100 cm^3$ de ce liquide possèdent une masse de $1,36 kg.$
\end{Box2}
%%_________________________Exercice 2 : _________________________Exercice

\begin{Box2}{Exercice 2 :}
   \begin{enumerate}
      \item La molécule du butane se compose de $4$ atomes de carbone $(C)$ et de $10$ atomes d’hydrogène $(H)$.
         \begin{enumerate}
               \item Donner la formule de cette molécule.
               \item Le butane est-il un corps pur composé ou simple ? Justifier la réponse.
         \end{enumerate}
      \item La masse d’un atome de carbone est $m_C = 1,99.10^{-23} g$ et la masse d’un atome d’hydrogène est $m_H=1,67.10^{-24}g.$
         \begin{enumerate}
               \item Calculer la masse d’une molécule de butane.
               \item Déterminer la masse de 4 moles de molécules de butane.
               \item Déterminer le nombre de moles de molécules de butane contenues dans un échantillon de masse 100 g.

         \end{enumerate}
   \end{enumerate}

\end{Box2}



%%_________________________Exercice 3 : _________________________Exercice
\begin{Box2}{Exercice 3 :}
Un pneu de voiture est gonflé à la température de 20,0°C sous la pression de 2,10 bar. Son volume intérieur, supposé constant, est de 30 L.
   \begin{enumerate}
         \item  Quel quantité d'air contient-il ?
         \item Après avoir roulé un certain temps, une vérification de la pression est effectuée: la pression est alors de 2,30 bar. Quelle est alors la température de l'air enfermé dans le pneu ? Exprimer le résultat dans l'échelle de  température usuelle.
         \item  Les valeurs de pression conseillées par les constructeurs pour un gonflage avec de l'air sont-elles différentes pour un gonflage à l'azote ?

   \end{enumerate}
Données: constante du gaz parfait, R= 8,314 SI

\end{Box2}

%%_________________________Exercice 4 : _________________________Exercice
\begin{Box2}{Exercice 4 :}
   \begin{enumerate}
      \item  L’alcool utilisé comme antiseptique local peut être considéré comme de l’éthanol $C_2H_6O$ pur de masse molaire $M = 46, 0g/mol$ et de masse volumique $\rho = 0, 780g/ml$ . Quelle quantité d’éthanol contient un flacon d’alcool pharmaceutique de volume $V = 250ml$ .
      \item L’éther éthylique de formule $C_4H_{10}O$ était jadis utilisé comme anesthésique. Sa masse molaire vaut $M = 74, 0g/mol$ et sa densité est égale à $d = 0, 710$ . On souhaite disposer d’une quantité $n = 0, 200mol$ . Quel volume faut-il prélever ?
   \end{enumerate}
   Donnée : masse volumique de l’eau : $\rho_{eau} = 1, 00g/ml$
\end{Box2}



% her you need to add sum extra exercices not forget porcentage




\begin{center}
   \Large{ \em{Exercices Supplémentaires}}
\end{center}
\begin{Box2}{Exercice 4 : }
Le vinaigre contient de l'acide éthanoïque de formule $CH_3CO_2H$. On réalise la réaction entre l'hydrogénocarbonate de sodium et un vinaigre de 6°. Il se forme du dioxyde de carbone et de l'eau.
\\1. Déterminer la concentration molaire en acide éthanoïque de ce vinaigre.
\\2. Écrire l'équation de la réaction.
\\3. On utilise un volume V=14mL de vinaigre. Sachant que l'acide éthanoïque est le réactif limitant, déterminer le volume de dioxyde de carbone formé dans les condition normales de température et de pression.
   \\\textbf{Donnée : Un vinaigre de x degrés contient x\% en masse d'acide éthanoïque et sa densité est égale à 1.}

\end{Box2}
%%_________________________Exercice 9 : _________________________Exercice

\begin{Box2}{Exercice 5 :}
Un flacon de déboucheur pour évier porte les indications suivantes :
\\Produit corrosif. Contient de l’hydroxyde de sodium (soude caustique). Solution à $20\%$.
\\Le pourcentage indiqué représente le pourcentage massique d’hydroxyde de sodium (NaOH) contenu dans le produit.
La densité du produit est d=1,2.
\\1. Calculer la masse d’hydroxyde de sodium contenu dans 500 mL de produit.
\\2. En déduire la concentration Co en soluté hydroxyde de sodium de la solution commerciale.
\\3. On désire préparer un volume V1 de solution S1 de déboucheur 20 fois moins concentré que la solution
commerciale.
\\3.1. Quelle est la valeur de la concentration C1 de la solution ?
\\3.2. Quelle est la quantité de matière d’hydroxyde de sodium contenu dans 250 mL de solution S1 ?
\\3.3. Quel volume de solution commerciale a-t-il fallu prélever pour avoir cette quantité de matière
d’hydroxyde de sodium ?
\end{Box2}

\begin{Box2}{Exercice 6}
Á température t = 20◦C et sous une pression $P = 1, 013 × 10^5Pa$ un hydrocarbure gazeux de formule
   $C_nH_{2n+2}$ a une densité par rapport à l’air $d = 2, 00$
\\1. Calculer le volume molaire des gaz dans les conditions étudiées .
\\2. Déterminer la masse molaire de l’hydrocarbure .
   \\3. En déduire sa formule brute . La masse volumique de l’air dans les conditions de l’étude $\rho{air} = 1, 21g/l$
\end{Box2}
\end{document}
