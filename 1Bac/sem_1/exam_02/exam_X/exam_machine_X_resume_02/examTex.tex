\documentclass[12pt]{article}
\usepackage[a4paper, margin=.30in]{geometry}
%\usepackage{array}
\usepackage{graphicx, subfig, wrapfig, makecell,fancybox }
\newcommand\headerMe[2]{\noindent{}#1\hfill#2}
\renewcommand \thesection{\Roman{section}}

\newcolumntype{M}[1]{>{\raggedright}m{#1}}



\begin{document}

\headerMe{Royaume du Maroc}{année scolaire \emph{2021-2022}}\\
\headerMe{Ministère de l'Éducation nationale, }{  Professeur :\emph{Zakaria Haouzan}}\\
\headerMe{du Préscolaire et des Sports}{Établissement : \emph{Lycée SKHOR qualifiant}}\\

\begin{center}
Devoir surveillé N°2 \\
1Bac Sciences Expérimentales\\
Durée 2h00
\\
    \vspace{.2cm}
\hrulefill
\Large{Fiche Pédagogique}
\hrulefill\\
\end{center}
%end Headerss------------------------


%__________________Chimie ______________________-
%%%%%%%+_+_+_+_+_+_+_+_+_Partie1
\section[A]{Introduction }
\hspace{0.5cm}Le programme d'études de la matière physique chimie vise à croître un ensemble de compétences visant à développer la personnalité de l'apprenant. Ces compétences peuvent être classées en Compétences transversales communes et Compétences qualitatives associées aux différentes parties du programme.
\section{cadre de référence }
 \hspace{0.5cm}L'épreuve a été réalisée en adoptant des modes proches à des situations d'apprentissages et des situations problèmes, qui permettent de compléter les connaissances et les compétences contenues dans les instructions pédagogiques et dans le programme de la matière physique chimie et aussi dans le cadre de référence de l'examen national. 
 \\Tout en respectant les rapports d'importance précisés dans les tableaux suivants :
 \begin{center}
\begin{tabular}{|c||c||c|}
\hline
    \textbf{Restitution des Connaissances} & \textbf{Application des Connaissances} & \textbf{Situation Problème }\\
    \hline
    $60\%$ & $20\%$ & $20\%$\\
    \hline
\end{tabular} 
\end{center}

\section{tableau de spécification}
 \begin{center}
\begin{tabular}{|c||c|c|c|c|}
\hline
    niveau d'habileté & \makecell{Restitution \\des Connaissances} &\makecell{Application \\des Connaissances} & \makecell{Situation Problème} & la somme \\
\hline
    \makecell{Travail et énergie\\ cinétique } & \makecell{\\30\%\\6pts\\36min}  & \makecell{\\10\%\\2pts\\12min}  &\makecell{\\10\%\\2pts\\12min } & \makecell{\\50\%\\10pts\\60min} \\\hline
    \makecell{Les grandeurs physiques \\liées à la \\quantité de matière }
    &\makecell{\\30\%\\6pts\\36min}  & \makecell{\\10\%\\2pts\\12min}  &\makecell{\\10\%\\2pts\\12min } & \makecell{\\50\%\\10pts\\60min} \\\hline
    
    &\makecell{60\%\\12pts\\72min}  & \makecell{20\%\\4pts\\24min}  &\makecell{20\%\\4pts\\24min } & \makecell{100\%\\20pts\\120min} \\\hline

\end{tabular} 
\end{center}

%Correction 


\newpage
\begin{center}
    \shadowbox{\bf{ Devoir surveillé $N^{\circ}$2 Semestre I} }
\end{center}
 \begin{center}

     \begin{tabular}{|c||c||c|}
    \hline
         \multicolumn{3}{||c||}{\bf{   \hfill  Chimie  \hfill (10pts)} }\\
         \hline
         \multicolumn{3}{||c||}{\bf{Partie 1 : La quantité de matière d’un échantillon \dotfill (10pts)} }\\
\hline
    \textbf{$N^{\circ}$Question } & \textbf{Réponse } & \textbf{Note }\\
    \hline
    $1.a$ &
         \makecell{ la quantité de matière contenue dans cette masse de Soufre \\$n(S) = \frac{m}{M} = 0.25 mol$
 }
    & $1pt$\\\hline
 %Q2
     $1.b$ &
         \makecell{ $N = n.N_A$ = $1505.10^{23}$
 }
    & $1pt$\\\hline  
 %Q3-a
     $2.a$ &
         \makecell{$n = \frac{m}{M} = \frac{\rho.V}{M} = \frac{d.\rho_{eau}.V}{M} = 1.17mol$
 }
    & $1pt$\\\hline  
 %Q3-b
     $2.b$ &
         \makecell{$m = M.n = 79g$ }
    & $1pt$\\\hline  
 %Q3-a
     $3.a$ &
         \makecell{ d =$ \frac{M}{29} = 1.1$
 }
    & $1pt$\\\hline  
 %Q3-b
     $3.b$ &
         \makecell{ PV = nRT\\
         n = 0.1mol  }
    & $1pt$\\\hline  
 %Q3-c
     $3.c$ &
         \makecell{n = $\frac{V}{V_m}$ \\ $V_m = 24L/mol$ }
    & $1pt$\\\hline  
 %Q3-d
     $3.d$ &
         \makecell{P'V' = nRT' \\ P' = 3045hPa }
    & $1pt$\\\hline  
 %Q4
     $4$ &
         \makecell{ n = $\frac{m}{M} = \frac{\rho.V}{M} =4.24mol $}
    & $2pt$\\\hline  


%Physique : 
    %Partie 1 : 
\end{tabular} 
\end{center}

\begin{center}
  \begin{tabular}{|c||c||c|}
    \hline
         \multicolumn{3}{||c||}{\bf{   \hfill  Physique  \hfill (10pts)} }\\
         \hline
         \multicolumn{3}{||c||}{\bf{Partie 1 : force motrice constante et énergie Cinétique \dotfill (6pts)} }\\
\hline
    \textbf{$N^{\circ}$Question } & \textbf{Réponse } & \textbf{Note }\\
    \hline
    $1.$ &
         \makecell{ l’énoncé du théorème de l’énergie cinétique : $\Delta{E_c} = \sum W(\vec{f})$ }
    & $0.5pt$\\\hline
 %Q2
 $2.$ &
         \makecell{En appliquant le théorème de l’énergie cinétique sur le \\corps S entre E et B qui est soumis à l’action des forces P et R\\
         $V_B = \sqrt{2gcos\alpha_0} = 5.09 m/s$
 }
    & $1.5pt$\\\hline
 %Q1.c
 $3.$ &
         \makecell{En appliquant le théorème de l’énergie cinétique sur le \\corps S entre A et B qui sera soumis à l’action \\des forces P et R et F :\\
         $F = \frac{m.V_B^2}{2.AB} = 4g.m.cos\alpha_0 = 364.4 N$
      }
    & $1.5pt$\\\hline
%Q2 : 
 $4$ &
         \makecell{ En appliquant le théorème de l’énergie cinétique sur le \\corps S entre D et E qui est soumis à l’action des forces P et R\\
         $V_D = \sqrt{2.g.r(cos\alpha_0 -cos\alpha )} = 3.31m/s$
 }
    & $1.5pt$\\\hline
%2.b
 $5$ &
         \makecell{En appliquant le théorème de l’énergie cinétique sur le \\corps S entre D et E qui est soumis à l’action des forces\\
         $V_B = \sqrt{2.g.r} = 5.4m/s$ donc $F = \frac{m.v^2_B}{2.AB} =400N $
 }
         
    & $1pt$\\\hline
      %Partie 2 : -----
\multicolumn{3}{||c||}{\bf{Partie 2 : Travail mécanique d’une machine  \dotfill (4pts)} }\\
\hline
%1
 $1.$ &
      \makecell{ $\Delta{E_c} = -\frac{1}{2}.{J_{\Delta}\omega^2_i} = -277J$}
    & $1pt$\\\hline
%2
 $2.$ &
      \makecell{ $ \mathcal{M}(\vec{f}) = f.r = 120N.m$}
    & $1pt$\\\hline
%3
 $3.$ &
      \makecell{ $ n = \frac{\Delta{E_c}}{\mathcal{M}(\vec{f}).2\pi} = 0.36$}
    & $2pt$\\\hline

  \end{tabular}
  \end{center}



\end{document}
