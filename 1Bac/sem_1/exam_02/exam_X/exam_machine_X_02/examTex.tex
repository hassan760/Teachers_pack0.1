\documentclass[12pt]{article}
\usepackage[a4paper, total={7.5in, 10in}]{geometry}
%\usepackage{array}
\usepackage{graphicx, subfig, wrapfig, fancyhdr, lastpage }
\newcommand\headerMe[2]{\noindent{}#1\hfill#2}
\usepackage[mathscr]{euscript}



\pagestyle{fancy}
\fancyhf{}

\cfoot{\em{Page \thepage \hspace{1pt} / \pageref{LastPage}}}
\begin{document}

\headerMe{Royaume du Maroc}{année scolaire \emph{2021-2022}}\\
\headerMe{Ministère de l'Éducation nationale, }{  Professeur :\emph{Zakaria Haouzan}}\\
\headerMe{du Préscolaire et des Sports}{Établissement : \emph{Lycée SKHOR qualifiant}}\\

\begin{center}
Devoir Surveillé  N°3 \\
    Filière 1Bac Sciences Expérimentales\\
Durée 2h00
\\
    \vspace{.2cm}
\hrulefill
\Large{Chimie 10pts}
\hrulefill\\

    %\emph{Les questions parties sont indépendantes}
\end{center}
%end Headerss------------------------
%__________________Chimie ______________________-
%%%%%%%+_+_+_+_+_+_+_+_+_Partie1

 \section*{Partie 1 : Les solutions électrolytiques \dotfill(10pts) }

Le chlorure de baryum $BaCl_2$ est un composé ionique constitué des ions chlorure et des ions baryum.

On fait dissoudre une masse $m=4,16g$ de chlorure de baryum dans un volume $V_1=200mL$ d’eau et on obtient une solution $S_1$ de concentration $C_1$.
\begin{enumerate}
    \item Quelle sont les étapes de dissolution du chlorure de baryum dans l’eau ?\dotfill(1pt)
    \item Ecrire l’équation de dissolution du chlorure de baryum dans l’eau.\dotfill(1pt)
    \item Donner l’expression de $C_1$ en fonction de m , M et $V_1$ puis calculer sa valeur. \dotfill(1pt)
        \item Déterminer l’expression de la concentration molaire effective de chacun des ions chlorure et des ions baryum dans la solution $S_1$ en fonction de $C_1$ puis calculer leurs valeurs.\dotfill(1pt)
            \item Déterminer l’expression de la quantité de matière de chacun des ions chlorure et des ions baryum dans la solution $S_1$ en fonction de $C_1$ et $V_1$ puis calculer leurs valeurs.\dotfill(1pt)
                \item On prépare une solution $S_2$ de volume $V_2=50mL$ de chlorure de calcium $CaCl_2$ de concentration $C_2=0,5mol/L$ en dissodissolvant une masse m’ de chlorure de calcium dans l’eau.
\begin{enumerate}
    \item Ecrire l’équation de dissolution puis déterminer l’expression de la concentration molaire effective de chacun des ions chlorure et des ions calcium en fonction de $C_2$ et calculer leurs valeurs.\dotfill(1pt)
    \item Déterminer l’expression de la quantité de matière de chacun des ions chlorure et des ions calcium dans la solution $S_2$ en fonction de $C_2$ et $V_2$ puis calculer leurs valeurs.\dotfill(1pt)
    \item Déterminer la valeur de la masse m’ utilisée pour préparer la solution $S_2$.\dotfill(1pt)
\end{enumerate}
\item On mélange la solution $S_1$ avec la solution $S_2$
    \begin{enumerate}
        \item Quels sont des ions présents dans le mélange obtenu.\dotfill(1pt)
        \item Déterminer l’expression de la concentration molaire effective de chacun des ions présents dans le mélange puis calculer leurs valeurs.\dotfill(1pt)
    \end{enumerate}
\end{enumerate}

On donne : M(Cl) = 35,5g/mol , M(Ba) = 137g/mol , M(Ca) = 40g/mol

%_____________________________________PHYSIque Partie 22222____________________________________________________________________________
\begin{center}
    \vspace{4cm}
\hrulefill
\Large{Physique 10pts}
\hrulefill\\
    \emph{Les parties sont indépendantes}
\end{center}
%end Headerss------------------------

 \section*{Partie 1 :  Travail et énergie mécanique  \dotfill(8 pts)}

Une pomme de masse m = 150g, accrochée à un pommier, se trouve à 3,0 m audessus du sol. Le sol est choisi comme référence des énergies potentielles de
pesanteur.
On donne g = 10 N/Kg

\begin{enumerate}
  \item Lorsque cette pomme est accrochée au pommier, quelle est :
  \begin{enumerate}
  \item son énergie cinétique ?\dotfill(1pt)
  \item son énergie potentielle de pesanteur ?\dotfill(1pt)
  \item son énergie mécanique ?\dotfill(1pt)
\end{enumerate}
  \item la pomme se détache et arrive au sol avec une vitesse de valeur V= 7,75 m/s. Calculer :
 \begin{enumerate}
  \item son énergie cinétique.\dotfill(1pt)
  \item son énergie potentielle de pesanteur.\dotfill(1pt)
  \item son énergie mécanique.\dotfill(1pt)
 \end{enumerate}
  \item  Quelles transformations énergétiques ont eu lieu au cours de cette chute ?\dotfill(1pt)

  \item  Quelle serait la hauteur de chute de cette pomme si elle arrivait au sol avec une
    vitesse de valeur V'=9,9 m/s\dotfill(1pt)

\end{enumerate}

\section*{Partie 2 :Mode de transfert d’énergie\dotfill (2pts)}

Si-Brahim a lancé une bille verticalement vers le haut à une altitude
h = 1,5m par rapport au sol, avec une vitesse $V_A= 10 m/s$.
On considère que le poids est la seule force appliquée à la
bille (chute libre).
On donne $g = 10 N/kg$.
Calculer en utilisant le théorème de l’énergie cinétique :

\begin{enumerate}
  \item La hauteur maximale atteinte par la bille.\dotfill(1pt)

  \item  La vitesse de la bille lorsqu’elle retombe sur le sol.\dotfill(1pt)
%_________________partie 2  : gravitation universelle :)

\end{enumerate}

\end{document}
