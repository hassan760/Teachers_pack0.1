\documentclass[12pt]{article}
\usepackage[a4paper, margin=.30in]{geometry}
%\usepackage{array}
\usepackage{fancybox}

\usepackage{graphicx, subfig, wrapfig, makecell }
\newcommand\headerMe[2]{\noindent{}#1\hfill#2}
\renewcommand \thesection{\Roman{section}}

\newcolumntype{M}[1]{>{\raggedright}m{#1}}




\begin{document}

\headerMe{Royaume du Maroc}{année scolaire \emph{2021-2022}}\\
\headerMe{Ministère de l'Éducation nationale, }{  Professeur :\emph{Zakaria Haouzan}}\\
\headerMe{du Préscolaire et des Sports}{Établissement : \emph{Lycée SKHOR qualifiant}}\\

\begin{center}
Devoir surveillé N°2 \\
1BAC Sciences Mathématiques\\
Durée 2h00
\\
    \vspace{.2cm}
\hrulefill
\Large{Fiche Pédagogique}
\hrulefill\\
\end{center}
%end Headerss------------------------


%__________________Chimie ______________________-
%%%%%%%+_+_+_+_+_+_+_+_+_Partie1
\section[A]{Introduction }
\hspace{0.5cm}Le programme d'études de la matière physique chimie vise à croître un ensemble de compétences visant à développer la personnalité de l'apprenant. Ces compétences peuvent être classées en Compétences transversales communes et Compétences qualitatives associées aux différentes parties du programme.
\section{cadre de référence }
 \hspace{0.5cm}L'épreuve a été réalisée en adoptant des modes proches à des situations d'apprentissages et des situations problèmes, qui permettent de compléter les connaissances et les compétences contenues dans les instructions pédagogiques et dans le programme de la matière physique chimie et aussi dans le cadre de référence de l'examen national. 
 \\Tout en respectant les rapports d'importance précisés dans les tableaux suivants :
 \begin{center}
\begin{tabular}{|c||c||c|}
\hline
    \textbf{Restitution des Connaissances} & \textbf{Application des Connaissances} & \textbf{Situation Problème }\\
    \hline
    $50\%$ & $25\%$ & $25\%$\\
    \hline
\end{tabular} 
\end{center}

\section{tableau de spécification}
 \begin{center}
\begin{tabular}{|c||c|c|c|c|}
\hline
    Niveau d'habileté & \makecell{Restitution \\des Connaissances} &\makecell{Application \\des Connaissances} & \makecell{Situation Problème} & la somme \\
\hline
    \makecell{Le travail : mode de \\transfert d’énergie
 } & \makecell{35\%\\7pts\\42min}  & \makecell{18\%\\4pts\\22min}  &\makecell{17\%\\3pts\\20min } & \makecell{70\%\\14pts\\84min} \\\hline
    \makecell{Les solutions \\électrolytiques\\ et les concentrations }
    &\makecell{9\%\\2pts\\12min}  & \makecell{4\%\\1pt\\4min}  &\makecell{4\%\\1pt\\4min } & \makecell{17\%\\4pts\\20min} \\\hline
    \makecell{Suivi d’une \\transformation \\chimique - Bilan \\de la matière }
&\makecell{6\%\\1pt\\10min}  & \makecell{3\%\\0.5pts\\3min}  &\makecell{3\%\\0.5pts\\3min } & \makecell{13\%\\2pt\\16min} \\\hline
 
&\makecell{50\%\\10pts\\64min}  & \makecell{25\%\\5.5pts\\29min}  &\makecell{25\%\\4.5pts\\27min } & \makecell{100\%\\20pts\\120min} \\\hline

\end{tabular} 
\end{center}

\newpage
\begin{center}
    \shadowbox{\bf{ Devoir surveillé $N^{\circ}$2 Semestre I} }
\end{center}
 \begin{center}

     \begin{tabular}{|c||c||c|}
    \hline
         \multicolumn{3}{||c||}{\bf{   \hfill  Chimie  \hfill (7pts)} }\\
         \hline
         \multicolumn{3}{||c||}{\bf{Partie 1 : Les solutions électrolytiques\dotfill (4pts)} }\\
\hline
    \textbf{$N^{\circ}$Question } & \textbf{Réponse } & \textbf{Note }\\
    \hline
    $1.$ &
         \makecell{la masse d’hydroxyde de sodium contenu dans 500mL $m_{NaOH} = d.\rho_{eau}.20\%.V$
         \\$m_{NaOH} = 120g$ 
 }
    & $1pt$\\\hline
 %Q2
     $2.$ &
     \makecell{ la concentration $C_0 = \frac{m_{NaOH}}{M(NaOH).V} = 4mol/L$  
 }
    & $1pt$\\\hline  
 %Q3-a
     $3.a$ &
         \makecell{ $C_1 = \frac{C_0}{20} = 0.2mol/L$
 }
    & $1pt$\\\hline  
 %Q3-b
     $3.b$ &
         \makecell{ $n(NaOH) = C_1.V_1 = 0.05mol$
 }
    & $0.5pt$\\\hline  
 %Q3-c
     $3.c$ &
         \makecell{ $ C_0.V_0 = C_1.V_1 $\\$V_0 = \frac{V_1}{20} = 12.5mL$
 }
    & $0.5pt$\\\hline  

%Partie 2 : 
         \multicolumn{3}{||c||}{\bf{Partie 2 : Suivi d’une transformation chimique\dotfill (2pts)}}\\
\hline
$1.$ &
         \makecell{\\ % table dont forget 
             \begin{tabular}{|c|c|c|c|c|c|}
    \hline
    \multicolumn{2}{|c|}{Equation de la réaction}& \multicolumn{4}{c|}{2CuO + C $\rightarrow$ 2Cu + $CO_2$}\\\hline
    états  & avancement& \multicolumn{4}{|c|}{quantité de Matière en mol}\\\hline
    Etat initial          &    0        &  12.38 &  1.4&  0              &  0 \\\hline
                 \makecell{Etat de \\transformation}&    $x$      & $ 12.38 - 2x$ & $ 1.4 - x$ & $ 2x$  & $x$ \\\hline
    Etat final            &    $x_{max}$& $ 12.38 - 2x_{max}$ & $1.4 - x_{max}$ & $2x_{max}$  & $x_{max}$ \\\hline
   % \cline{2-4}\
\end{tabular}
         \\$\; $ }  
    & $1pt$\\\hline
 %Q2
   $2$ &
         \makecell{ l’avancement maximal $x_{max} = 1.4mol$ et le réactif limitant le carbone C(s) 
 }
    & $0.5pt$\\\hline  
 %Q3
   $3$ &
         \makecell{ bilan de matière dans l’état final :\\ $n_f(CuO) = 9.58 mol$ et $n_f(C) = 0 mol$ et $n_f(Cu) = 2.8 mol$, $n_f(CO_2) = 1.4mol$ 
 }
    & $0.5pt$\\\hline  
%Physique : 
    %Partie 1 : 
\end{tabular} 
\end{center}

\begin{center}
  \begin{tabular}{|c||c||c|}
    \hline
         \multicolumn{3}{||c||}{\bf{   \hfill  Physique  \hfill (13pts)} }\\
         \hline
         \multicolumn{3}{||c||}{\bf{Partie 1 : Travail mécanique et énergie \dotfill (11pts)} }\\
\hline
    \textbf{$N^{\circ}$Question } & \textbf{Réponse } & \textbf{Note }\\
    \hline
    $1.a$ &
         \makecell{
             Bilan des forces : $\vec{P}$ poids du corps.\\ et $\vec{R}$ : réaction du plan , le contact se fait sans frottement.\\
         $\vec{F}$ : la force motrice .
     }
    & $1pt$\\\hline
 %Q2
 $1.b$ &
         \makecell{ $\Delta{E_c} = \sum W(\vec{f})$ }
    & $1pt$\\\hline
 %Q1.c
 $1.c$ &
         \makecell{En appliquant le théorème de l’énergie cinétique sur le corps S entre A et B \\
      $ \Delta{E_c}_{A \rightarrow B} =W(\vec{P})_{A \rightarrow B}+W(\vec{R})_{A \rightarrow B}  W(\vec{F})_{A \rightarrow B}$ \\
      ${E_c}_B =-mg.sin\alpha + F.AB $\\
      $F =\frac{{E_c}_B + mg.AB.sin\alpha}{AB}  = 5.2N$
      }
    & $1pt$\\\hline
%Q2 : 
 $2.a$ &
         \makecell{ variation de l’énergie potentielle de pesanteur du corps S entre B et C :\\
         $\Delta{E_{pp}}_{B \rightarrow C} = {E_{pp}}_C - {E_{pp}}_B = mg(z_c - z_B) = mg.BC.sin\alpha$}
    & $1pt$\\\hline
%2.b
 $2.b$ &
         \makecell{  
$\Delta{E_{m}}_{B \rightarrow C} = \frac{1}{2}m(v_c^2 - v_B^2) + mg.BC.sin\alpha$}
         
    & $1pt$\\\hline
%2.c
 $2.c$ &
         \makecell{ Le contact se fait avec frottement sur le trajet BC , donc la variation de \\l’énergie mécanique est égale au travail de la force de frottement. \\ 
      $\Delta{E_{m}}_{B \rightarrow C} = W(\vec{f})_{B \rightarrow C }$\\
      donc : $f = -\frac{\Delta{E_m}}{BC} = 2.8N$}
         
    & $1pt$\\\hline

%3.a
 $3.a$ &
         \makecell{${E_m}_c = E_c + E_p = \frac{1}{2}mv_c^2 + mg(z_C - z_B) =  \frac{1}{2}mv_c^2 + mg.BCsin\alpha  = 1.54J$ }
    & $1pt$\\\hline
%3.b
 $3.b$ &
      \makecell{ ${E_m}_m = E_c + E_p = \frac{1}{2}mv_m^2 + mg(z_m - z_B) = mg(z_m - z_B) $\\$E_m = mg(BC.sin\alpha + r(cos\alpha - cos(\alpha + \theta))) $}
    & $2pt$\\\hline
%3.c
 $3.c$ &
      \makecell{$\theta =19.1^{\circ}$ }
    & $2pt$\\\hline


      %Partie 2 : -----
\multicolumn{3}{||c||}{\bf{Partie 2 : Mode de transfert d’énergie \dotfill (3pts)} }\\
\hline
%1
 $1$ &
      \makecell{ $Z_{max} = 6.5m$}
    & $1.5pt$\\\hline
%2
 $1$ &
      \makecell{ $V_f = 12.6m/s$}
    & $1.5pt$\\\hline







  \end{tabular}
  \end{center}


\end{document}
