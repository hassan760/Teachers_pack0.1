\documentclass[12pt]{article}
\usepackage[a4paper, margin=.30in]{geometry}
%\usepackage{array}
\usepackage{fancybox}

\usepackage{graphicx, subfig, wrapfig, makecell,mathrsfs }
\newcommand\headerMe[2]{\noindent{}#1\hfill#2}
\renewcommand \thesection{\Roman{section}}

\newcolumntype{M}[1]{>{\raggedright}m{#1}}




\begin{document}

\headerMe{Royaume du Maroc}{année scolaire \emph{2021-2022}}\\
\headerMe{Ministère de l'Éducation nationale, }{  Professeur :\emph{Zakaria Haouzan}}\\
\headerMe{du Préscolaire et des Sports}{Établissement : \emph{Lycée SKHOR qualifiant}}\\

\begin{center}
Devoir surveillé N°1 \\
 Filière Tronc Commun Scientifique \\
Durée 1h30
\\
    \vspace{.2cm}
\hrulefill
\Large{Fiche Pédagogique}
\hrulefill\\
\end{center}
%end Headerss------------------------


%__________________Chimie ______________________-
%%%%%%%+_+_+_+_+_+_+_+_+_Partie1
\section[A]{Introduction }
\hspace{0.5cm}Le programme d'études de la matière physique chimie vise à croître un ensemble de compétences visant à développer la personnalité de l'apprenant. Ces compétences peuvent être classées en Compétences transversales communes et Compétences qualitatives associées aux différentes parties du programme.
\section{cadre de référence }
 \hspace{0.5cm}L'épreuve a été réalisée en adoptant des modes proches à des situations d'apprentissages et des situations problèmes, qui permettent de compléter les connaissances et les compétences contenues dans les instructions pédagogiques et dans le programme de la matière physique chimie et aussi dans le cadre de référence de l'examen national. 
 \\Tout en respectant les rapports d'importance précisés dans les tableaux suivants :
 \begin{center}
\begin{tabular}{|c||c||c|}
\hline
    \textbf{Restitution des Connaissances} & \textbf{Application des Connaissances} & \textbf{Situation Problème }\\
    \hline
    $60\%$ & $30\%$ & $10\%$\\
    \hline
\end{tabular} 
\end{center}

\section{tableau de spécification}
 \begin{center}
\begin{tabular}{|c||c|c|c|c|}
\hline
    Niveau d'habileté & \makecell{Restitution \\des Connaissances} &\makecell{Application \\des Connaissances} & \makecell{Situation Problème} & la somme \\
\hline
    \makecell{Équilibre d’un corps\\ solide soumis à trois \\forces non
parallèles
 } & \makecell{20\%\\4pts\\20min\\3q}  & \makecell{10\%\\3pts\\10min\\2q}  &\makecell{3\%\\0pt\\0min } & \makecell{33\%\\7pts\\30min\\5Q} \\\hline
    \makecell{Équilibre d’un corps \\solide en rotation autour\\ d’un axe
fixe }
    &\makecell{20\%\\4pts\\20min\\3q}  & \makecell{10\%\\2pt\\10min\\2q}  &\makecell{2\%\\0pt\\0min } & \makecell{32\%\\6pts\\30min\\6Q} \\\hline
    \makecell{ Classification périodique \\des éléments chimiques}
&\makecell{22\%\\4pt\\20min\\3q}  & \makecell{10\%\\3pts\\10min\\2q}  &\makecell{3\%\\0pts\\0min } & \makecell{35\%\\7pt\\30min\\5q} \\\hline
 
&\makecell{60\%\\12pts\\60min\\9q}  & \makecell{30\%\\8pts\\30min\\6q}  &\makecell{10\%\\0pts\\0min } & \makecell{100\%\\20pts\\90min} \\\hline

\end{tabular} 
\end{center}

\newpage
\begin{center}
    \shadowbox{\bf{ Devoir surveillé $N^{\circ}$1 Semestre II} }
\end{center}
 \begin{center}

     \begin{tabular}{|c||c||c|}
    \hline
         \multicolumn{3}{||c||}{\bf{   \hfill  Chimie  \hfill (7pts)} }\\
         \hline
         \multicolumn{3}{||c||}{\bf{Partie 1  : La quantité de matière\dotfill (7pts)} }\\
\hline
    \textbf{$N^{\circ}$Question } & \textbf{Réponse } & \textbf{Note }\\
    \hline
    $1.$ &
         \makecell{
             $M(C_8H_10N_4O_2) = 194g/mol$
         }
    & $1pt$\\\hline
 %Q2
     $2.$ &
     \makecell{quantité de matière de caféine : $n = 4,1.10^-4 mol$
 }
    & $1pt$\\\hline  
 %Q3
     $3$ &
         \makecell{
            $N=2,5.10^20$ molécules de caféine dans la tasse.
         }
    & $1pt$\\\hline  
 %Q4
     $4$ &
         \makecell{
             On peut boire 7 tasses sans risques.
         }
    & $2pt$\\\hline  
 %Q3-c
     $5$ &
         \makecell{
             $ n=0,13 mol $
 }
    & $2pt$\\\hline  

%Physique : 
    %Partie 1 : 
\end{tabular} 
\end{center}

\begin{center}
  \begin{tabular}{|c||c||c|}
    \hline
         \multicolumn{3}{||c||}{\bf{   \hfill  Physique  \hfill (13pts)} }\\
         \hline
         \multicolumn{3}{||c||}{\bf{Partie 1 :Le courant électrique continu \dotfill (7pts)} }\\
\hline
    \textbf{$N^{\circ}$Question } & \textbf{Réponse } & \textbf{Note }\\
    \hline
    $1.a$ &
         \makecell{
             $N=19.10^{19}$ 
         }
    & $1pt$\\\hline
 %Q2
 $1.b$ &
         \makecell{
             Déduisons la valeur de l'intensité du courant I1 qui \\traverse la lampe L1. :$I_1=0.50A$
         }
    & $1pt$\\\hline
 %Q1.c
 $2.a$ &
         \makecell{
             L'ampèremètre A comporte 100 divisions et possède les calibres suivant :
             \\Le calibre le plus adapté pour la mesure de l'intensité I1 est le plus petit calibre \\supérieur à l'intensité du courant à mesurer, \\c'est-à-dire le calibre 1A
      }
      & $1pt$\\\hline
%Q2 : 
 $b.$ &
         \makecell{
             Déterminons la division n devant laquelle l'aiguille de \\l'ampèremètre s'arrête n=50 divisions
         }
    & $1pt$\\\hline
%5
 $3.a$ &
         \makecell{
             Les points qui sont considérés comme des nœuds sont F  et  C
         }
         
    & $1pt$\\\hline
 $3.b/c$ &
         \makecell{
             Indiquons le sens du courant dans chaque branche.
             \\Donc, $I=0.8A$ traverse la lampe $L_2$  et  $I'=0.30A$ traverse les lampes $L_3$  et $L_4$
         }
         
    & $1pt$\\\hline


      %Partie 2 : -----
\multicolumn{3}{||c||}{\bf{Partie 2 : La Mesure de l’intensité du courant éléctrique\dotfill (6pts)} }\\
\hline
%1
 $1.$ &
      \makecell{
          le sens des différents courants électriques dans
les branches du circuit. .
      }
    & $2pt$\\\hline
%2
 $2.$ &
      \makecell{
Compléter le tableau des intensités. .\\
          \begin{tabular}{ | l | l | l | l |l|l|l|l|l|}
    \hline
              Branche       & NP & PA & AB  & BN & PC& CD & DN & AN  \\\hline
              Intensité (A) & 3  &  2 & 0.5 & 1  &  1&1   &1   &1.5       \\ \hline
    \end{tabular}
      }
    & $2pt$\\\hline
%2.c
 $3.$ &
         \makecell{
             Compléter les tableaux suivants : 
            \\ fig1 : 
            \\C =  
            \\n=
            \\n0=
         }
         
    & $1pt$\\\hline

%%3.a
 %$4$ &
         %\makecell{
             %$F = \frac{m.g.OA.cos\alpha}{2.OH} = 147.05N$}
    %& $2pt$\\\hline
%%3.b
 %$3.b$ &
      %\makecell{ ${E_m}_m = E_c + E_p = \frac{1}{2}mv_m^2 + mg(z_m - z_B) = mg(z_m - z_B) $\\$E_m = mg(BC.sin\alpha + r(cos\alpha - cos(\alpha + \theta))) $}
    %& $2pt$\\\hline
%%3.c
 %$3.c$ &
      %\makecell{$\theta =19.1^{\circ}$ }
    %& $2pt$\\\hline








  \end{tabular}
  \end{center}


\end{document}
