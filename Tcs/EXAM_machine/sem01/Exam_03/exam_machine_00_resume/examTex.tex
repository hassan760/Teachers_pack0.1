\documentclass[12pt]{article}
\usepackage[a4paper, margin=.30in]{geometry}
%\usepackage{array}
\usepackage{fancybox}

\usepackage{graphicx, subfig, wrapfig, makecell,mathrsfs }
\newcommand\headerMe[2]{\noindent{}#1\hfill#2}
\renewcommand \thesection{\Roman{section}}

\newcolumntype{M}[1]{>{\raggedright}m{#1}}




\begin{document}

\headerMe{Royaume du Maroc}{année scolaire \emph{2022-2023}}\\
\headerMe{Ministère de l'Éducation nationale, }{  Professeur :\emph{Zakaria Haouzan}}\\
\headerMe{du Préscolaire et des Sports}{Établissement : \emph{Lycée SKHOR qualifiant}}\\

\begin{center}
Devoir surveillé N°1 \\
 Filière Tronc Commun Scientifique \\
Durée 2h00
\\
    \vspace{.2cm}
\hrulefill
\Large{Fiche Pédagogique}
\hrulefill\\
\end{center}
%end Headerss------------------------


%__________________Chimie ______________________-
%%%%%%%+_+_+_+_+_+_+_+_+_Partie1
\section[A]{Introduction }
\hspace{0.5cm}Le programme d'études de la matière physique chimie vise à croître un ensemble de compétences visant à développer la personnalité de l'apprenant. Ces compétences peuvent être classées en Compétences transversales communes et Compétences qualitatives associées aux différentes parties du programme.
\section{cadre de référence }
 \hspace{0.5cm}L'épreuve a été réalisée en adoptant des modes proches à des situations d'apprentissages et des situations problèmes, qui permettent de compléter les connaissances et les compétences contenues dans les instructions pédagogiques et dans le programme de la matière physique chimie et aussi dans le cadre de référence de l'examen national. 
 \\Tout en respectant les rapports d'importance précisés dans les tableaux suivants :
 \begin{center}
\begin{tabular}{|c||c||c|}
\hline
    \textbf{Restitution des Connaissances} & \textbf{Application des Connaissances} & \textbf{Situation Problème }\\
    \hline
    $60\%$ & $30\%$ & $10\%$\\
    \hline
\end{tabular} 
\end{center}

\section{tableau de spécification}
 \begin{center}
\begin{tabular}{|c||c|c|c|c|}
\hline
    Niveau d'habileté & \makecell{Restitution \\des Connaissances} &\makecell{Application \\des Connaissances} & \makecell{Situation Problème} & la somme \\
\hline
    \makecell{Gravitation universelle
 } & \makecell{18\%\\3,5pts - 3Q}  & \makecell{9\%\\2pts - 2Q}  &\makecell{3\%\\0,5pts-1Q} & \makecell{30\%\\6pts - 6Q} \\\hline
    \makecell{Exemples d’actions mécaniques}
    &\makecell{7\%\\1,5pts - 1Q}  & \makecell{3\%\\0,75pt - 1Q}  &\makecell{2\% } & \makecell{12\%\\2,25pts\\ 2Q} \\\hline
    \makecell{Les espèces chimiques }
&\makecell{7\%\\1,5pts - 1Q}  & \makecell{4\%\\1,25pts - 1Q}  &\makecell{1\%} & \makecell{12\%\\2,75pt\\ 2Q} \\\hline

     \makecell{Extraction , séparation , \\et identification \\d’espèces chimiques }
&\makecell{14\%\\2,75pt - 3q}  & \makecell{10\%\\3pts - 2q}  &\makecell{3\% } & \makecell{35\%\\7pt\\30min\\5q} \\\hline
 
    \makecell{Synthèse d’espèces chimiques }
&\makecell{22\%\\4pt - 3q}  & \makecell{10\%\\3pts - 2q}  &\makecell{3\% } & \makecell{35\%\\7pt\\30min\\5q} \\\hline


&\makecell{60\%\\12pts - 9q}  & \makecell{30\%\\8pts - 6q}  &\makecell{10\% } & \makecell{100\%\\20pts\\90min} \\\hline

\end{tabular} 
\end{center}

\newpage
\begin{center}
    \shadowbox{\bf{ Devoir surveillé $N^{\circ}$1 Semestre II} }
\end{center}
 \begin{center}

     \begin{tabular}{|c||c||c|}
    \hline
         \multicolumn{3}{||c||}{\bf{   \hfill  Chimie  \hfill (7pts)} }\\
         \hline
         \multicolumn{3}{||c||}{\bf{Partie 1  : Classification périodique des éléments chimiques \dotfill (7pts)} }\\
\hline
    \textbf{$N^{\circ}$Question } & \textbf{Réponse } & \textbf{Note }\\
    \hline
    $1.$ &
         \makecell{
           la structure électronique :  $(k)^2(L)^8(M)^1$    
         }
    & $1pt$\\\hline
 %Q2
     $2.$ &
     \makecell{l’élément chimique correspondant :
         la 3eme période couche (M) Groupe (1) 
 }
    & $1pt$\\\hline  
 %Q3
     $3$ &
         \makecell{
             Z = 11 et $_{11}Na$
         }
    & $2pt$\\\hline  
 %Q4
     $4$ &
         \makecell{
             ion monoatomique : $Na^+$
         }
    & $1pt$\\\hline  
 %Q3-c
     $5$ &
         \makecell{
             la famille des métaux alcalins (H, Li, Na)
 }
    & $2pt$\\\hline  

%Physique : 
    %Partie 1 : 
\end{tabular} 
\end{center}

\begin{center}
  \begin{tabular}{|c||c||c|}
    \hline
         \multicolumn{3}{||c||}{\bf{   \hfill  Physique  \hfill (13pts)} }\\
         \hline
         \multicolumn{3}{||c||}{\bf{Partie 1 :Équilibre d’un corps solide soumis à trois forces non
parallèles \dotfill (7pts)} }\\
\hline
    \textbf{$N^{\circ}$Question } & \textbf{Réponse } & \textbf{Note }\\
    \hline
    $1.$ &
         \makecell{
             Bilan des forces : $\vec{P}$ poids du solide.\\ et $\vec{T}$ : la force appliquée par le ressort (R).\\
         $\vec{F}$ : la force appliquée par le fil(F) .\\
         les représenter sur la figure
     }
    & $1pt$\\\hline
 %Q2
 $2.$ &
         \makecell{
             les droites d’action des trois forces sont coplanaires et concourantes.\\
             $\sum{\vec{f_{ext}}} = \vec{0}$
         }
    & $1pt$\\\hline
 %Q1.c
 $3.$ &
         \makecell{les expressions des coordonnées de chacune des forces dans \\le repére (O, x, y) en fonction de
leurs intensités
\\$P \{P_x = 0 ;; P_y = P\}$
      \\$F \{F_x = F ;; F_y = 0\}$
      \\$T \{ T_x = -Tsin\alpha ;; T_y = -Tcos\alpha\}$
      }
      & $2pt$\\\hline
%Q2 : 
 $4.$ &
         \makecell{l’expression de la tension T du ressort \\en fonction de m, g et $\alpha$ $T = \frac{m.g}{cos\alpha}$  }
    & $1pt$\\\hline
%5
 $5.$ &
         \makecell{
             T = 2.3N et $\Delta{l} = 0.05m$
         }
         
    & $2pt$\\\hline

      %Partie 2 : -----
\multicolumn{3}{||c||}{\bf{Partie 2 :Équilibre d’un corps solide en rotation autour d’un axe
fixe \dotfill (6pts)} }\\
\hline
%1
 $1.$ &
      \makecell{
          Bilan des forces : $\vec{P}$ poids du panneau .\\ et $\vec{R}$ :  réaction de l'axe ($\Delta{}$).\\
         $\vec{F}$ : perpendiculaire au panneau en (H) .\\
         les représenter sur la figure
      }
    & $2pt$\\\hline
%2
 $2.$ &
      \makecell{théorème des moments. }
    & $1pt$\\\hline
%2.c
 $3.$ &
         \makecell{
l’expression du moment de chaque force appliquée sur le panneau :
\\ $\mathscr{M}(F) = -F.OH et \mathscr{M}(P) = +P.OG.cos\alpha$
         }
         
    & $1pt$\\\hline

%3.a
 $4$ &
         \makecell{
             $F = \frac{m.g.OA.cos\alpha}{2.OH} = 147.05N$}
    & $2pt$\\\hline
%%3.b
 %$3.b$ &
      %\makecell{ ${E_m}_m = E_c + E_p = \frac{1}{2}mv_m^2 + mg(z_m - z_B) = mg(z_m - z_B) $\\$E_m = mg(BC.sin\alpha + r(cos\alpha - cos(\alpha + \theta))) $}
    %& $2pt$\\\hline
%%3.c
 %$3.c$ &
      %\makecell{$\theta =19.1^{\circ}$ }
    %& $2pt$\\\hline








  \end{tabular}
  \end{center}


\end{document}
