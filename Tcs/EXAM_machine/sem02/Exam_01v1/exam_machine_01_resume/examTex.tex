\documentclass[12pt]{article}
\usepackage[a4paper, margin=.30in]{geometry}
%\usepackage{array}
\usepackage{fancybox}

\usepackage{graphicx, subfig, wrapfig, makecell,mathrsfs }
\newcommand\headerMe[2]{\noindent{}#1\hfill#2}
\renewcommand \thesection{\Roman{section}}

\newcolumntype{M}[1]{>{\raggedright}m{#1}}




\begin{document}

\headerMe{Royaume du Maroc}{année scolaire \emph{2021-2022}}\\
\headerMe{Ministère de l'Éducation nationale, }{  Professeur :\emph{Zakaria Haouzan}}\\
\headerMe{du Préscolaire et des Sports}{Établissement : \emph{Lycée SKHOR qualifiant}}\\

\begin{center}
Devoir surveillé N°2 \\
 Filière Tronc Commun Scientifique \\
Durée 2h00
\\
    \vspace{.2cm}
\hrulefill
\Large{Fiche Pédagogique}
\hrulefill\\
\end{center}
%end Headerss------------------------


%__________________Chimie ______________________-
%%%%%%%+_+_+_+_+_+_+_+_+_Partie1
\section[A]{Introduction }
\hspace{0.5cm}Le programme d'études de la matière physique chimie vise à croître un ensemble de compétences visant à développer la personnalité de l'apprenant. Ces compétences peuvent être classées en Compétences transversales communes et Compétences qualitatives associées aux différentes parties du programme.
\section{cadre de référence }
 \hspace{0.5cm}L'épreuve a été réalisée en adoptant des modes proches à des situations d'apprentissages et des situations problèmes, qui permettent de compléter les connaissances et les compétences contenues dans les instructions pédagogiques et dans le programme de la matière physique chimie et aussi dans le cadre de référence de l'examen national. 
 \\Tout en respectant les rapports d'importance précisés dans les tableaux suivants :
 \begin{center}
\begin{tabular}{|c||c||c|}
\hline
    \textbf{Restitution des Connaissances} & \textbf{Application des Connaissances} & \textbf{Situation Problème }\\
    \hline
    $60\%$ & $30\%$ & $10\%$\\
    \hline
\end{tabular} 
\end{center}

\section{tableau de spécification}
 \begin{center}
\begin{tabular}{|c||c|c|c|c|}
\hline
    Niveau d'habileté & \makecell{Restitution \\des Connaissances} &\makecell{Application \\des Connaissances} & \makecell{Situation Problème} & la somme \\
\hline
    \makecell{Équilibre d’un corps\\ solide soumis à trois \\forces non
parallèles
 } & \makecell{20\%\\4pts\\20min\\3q}  & \makecell{10\%\\3pts\\10min\\2q}  &\makecell{3\%\\0pt\\0min } & \makecell{33\%\\7pts\\30min\\5Q} \\\hline
    \makecell{Équilibre d’un corps \\solide en rotation autour\\ d’un axe
fixe }
    &\makecell{20\%\\4pts\\20min\\3q}  & \makecell{10\%\\2pt\\10min\\2q}  &\makecell{2\%\\0pt\\0min } & \makecell{32\%\\6pts\\30min\\6Q} \\\hline
    \makecell{ Classification périodique \\des éléments chimiques}
&\makecell{22\%\\4pt\\20min\\3q}  & \makecell{10\%\\3pts\\10min\\2q}  &\makecell{3\%\\0pts\\0min } & \makecell{35\%\\7pt\\30min\\5q} \\\hline
 
&\makecell{60\%\\12pts\\60min\\9q}  & \makecell{30\%\\8pts\\30min\\6q}  &\makecell{10\%\\0pts\\0min } & \makecell{100\%\\20pts\\90min} \\\hline

\end{tabular} 
\end{center}

\newpage
\begin{center}
    \shadowbox{\bf{ Devoir surveillé $N^{\circ}$1 Semestre II} }
\end{center}
 \begin{center}

     \begin{tabular}{|c||c||c|}
    \hline
         \multicolumn{3}{||c||}{\bf{   \hfill  Chimie  \hfill (7pts)} }\\
         \hline
         \multicolumn{3}{||c||}{\bf{Partie 1  : Les pluies acides \dotfill (4pts)} }\\
\hline
    \textbf{$N^{\circ}$Question } & \textbf{Réponse } & \textbf{Note }\\
    \hline
    $1.$ &
         \makecell{la masse molaire moléculaire du trioxyde de soufre:
             $M(SO_3) = 80g/mol$
         }
    & $1pt$\\\hline
 %Q2
     $2.$ &
     \makecell{
         La masse de trioxyde de soufre dans la ville.
         \\$m = 2.56.10^{-4}g$
 }
    & $1pt$\\\hline  
 %Q3
     $3.$ &
         \makecell{
         L’aire de cette ville est considéré comme pollué
         }
    & $2pt$\\\hline  
    \multicolumn{3}{||c||}{\bf{Partie 2  : La quantité de matière du cholestérol . \dotfill (3pts)} }\\
\hline
    \textbf{$N^{\circ}$Question } & \textbf{Réponse } & \textbf{Note }\\
    \hline
    $1.$ &
         \makecell{
             l’expression de masse molaire\\ $M(C_xH_{2x-8}O) = x.M(C) + (2x-8).M(H)+M(O)$
         }
    & $1pt$\\\hline
    %Q2
     $2.$ &
         \makecell{
             $x = 27$ donc la formule brute du cholestérol.$C_{27}H_{46}O$
         }
    & $1pt$\\\hline  
 %Q3-c
     $3.$ &
         \makecell{Ce personne est malade car la masse du cholestérol $m = 2.5g$ }
    & $1pt$\\\hline  

%Physique : 
    %Partie 1 : 
\end{tabular} 
\end{center}

\begin{center}
  \begin{tabular}{|c||c||c|}
    \hline
         \multicolumn{3}{||c||}{\bf{   \hfill  Physique  \hfill (13pts)} }\\
         \hline
         \multicolumn{3}{||c||}{\bf{Partie 1 :La Mesure de l’intensité du courant éléctrique \dotfill (7pts)} }\\
\hline
    \textbf{$N^{\circ}$Question } & \textbf{Réponse } & \textbf{Note }\\
    \hline
    $1.$ &
         \makecell{La quantité d’électricité Q=3.2C
         }
    & $1pt$\\\hline
 %Q2
 $2.$ &
         \makecell{
             le nombre d’électrons (n) traversant une section du conducteur\\ $N = 2.10^{19}$
         }
    & $1pt$\\\hline
 %Q1.c
 $3.a$ &
         \makecell{
             Un ampèremètre se branche en série dans le circuit.\\ Cela veut dire qu'il faut ouvrir le circuit à l'endroit où \\l'on souhaite mesurer l'intensité et placer l'ampèremètre entre \\les deux bornes créées par cette ouverture du circuit.
      }
      & $1pt$\\\hline
%Q2 : 
 $3.b.$ &
         \makecell{
             Le calibre le plus proche et supérieur à la valeur mesurée est\\ le calibre 500mA.Il faut donc le choisir pour avoir\\ la valeur de courant  la plus précise sans endommager l'ampèremètre.
         }
    & $1pt$\\\hline
%5
 $3.c$ &
         \makecell{
             l’aiguille de l’ampèremètre fixera sur la graduation \\
             $N = I.\frac{N_0}{C} = 60$
         }
         
    & $1pt$\\\hline
 $3.d$ &
         \makecell{
             l’incertitude absolue sur la mesure de l’intensité.\\
             $\Delta{I} = C.\frac{x}{100} = 7.5mA$\\
             l’incertitude relative : $\frac{\Delta{I}}{I} = 2.5$\%
         }
         
    & $2pt$\\\hline


      %Partie 2 : -----
\multicolumn{3}{||c||}{\bf{Partie 2 : Le courant électrique continu \dotfill (6pts)} }\\
\hline
%1
 $1.$ &
      \makecell{
      Les deux points A et B sont des Nœuds
      }
    & $1pt$\\\hline
%2
 $2.$ &
      \makecell{
      le sens des courants manquants dans chaque branche du circuit.
      }
    & $1pt$\\\hline
%2.c
 $3.$ &
         \makecell{
             L'intensité du courant électrique I = 8.5A
         }
         
    & $1pt$\\\hline
%2.c
 $4.$ &
         \makecell{la relation entre I, $I_1$ , $I_2$ et $I_3$ : I = $I_1 + I_2 + I_3$\\
la relation entre $I_1$ , $I_2$ et $I_4$ : $I_1 + I_2 = I_4$
\\la relation entre $I_3, I_4 , I_5$ et $I_6$ : $I_3 + I_4 = I_6 + I_5$}
&$2pt$\\\hline
 $5.$ &
         \makecell{
             $I_1=3.5A$ et $I_4 = 5.5A$ et $I_5= 7A$
         }
&$1pt$\\\hline









  \end{tabular}
  \end{center}


\end{document}
