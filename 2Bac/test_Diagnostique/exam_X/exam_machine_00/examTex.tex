\documentclass[12pt]{article}
\usepackage[a4paper, total={7.66in,  11.1in}]{geometry}
%\usepackage{array}
\usepackage{graphicx, subfig, wrapfig ,multicol, color,mathrsfs,mhchem, chemfig,fancyhdr,lastpage }
\setlength{\columnseprule}{1pt}
\def\columnseprulecolor{\color{blue}}
\usepackage[mathscr]{euscript}
\newcommand\headerMe[2]{\noindent{}#1\hfill#2}

\pagestyle{fancy}
\fancyhf{}
\cfoot{\vspace{-1cm}\em{Page:\thepage / \pageref{LastPage}}}

\begin{document}
\headerMe{Royaume du Maroc}{Année Scolaire \emph{2023-2024}}\\
\headerMe{Ministère de l'Éducation Nationale, }{  Professeur :\emph{Zakaria Haouzan}}\\
\headerMe{du Préscolaire et des Sports}{Établissement : \emph{Lycée SKHOR qualifiant}}\\
\begin{center}
	\vspace{-0.5cm}
Evaluation Diagnostique \\
Durée 1h45
\\
 %   \vspace{.2cm}
%\hrulefill
%\Large{Chimie 6pts}
%\hrulefill\\
 %   \emph{Les deux parties sont indépendantes}
\end{center}
%end Headerss------------------------
\begin{center}
	\vspace{-0.2cm}
	\textbf{ Prénom.......................................  Nom.......................................}
	
	\vspace{0.5cm}
	\textbf{ Date.........................................  classe:................................... Note: }
		


\end{center}
\begin{center}

	\vspace{-0.75cm}
\underline{Consignes aux élèves : L’évaluation comporte 4 Parties: Mécanique, Electrodynamique ; Optique et Chimie}

	\hrulefill
\textbf{Physique 70\%}
\hrulefill
\end{center}
%__________________Chimie ______________________-
%%%%%%%+_+_+_+_+_+_+_+_+_Partie1

\vspace{-1cm}
\begin{multicols}{2}
    [
        \section*{Partie 1 : Mécanique \textit{Choisir les bonnes réponses.}}
] 

\begin{enumerate}
  \item La relation entre la vitesse linéaire et la vitesse angulaire est :
	  \vspace{-0.3cm}
      \begin{enumerate}
          \item $V = R.\omega$\sloppy
          \item $\omega = R.V$\nolinebreak
          \item $R = V.\omega$
      \end{enumerate}

  \item  Unité de la puissance d’une force est :
      \begin{enumerate}
          \item Joule 
		  \item Newton  
		  \item Watt
      \end{enumerate}


 \item  Un solide est animé d’un mouvement de translation rectiligne uniforme. Il est soumis à deux
forces constantes :
      \begin{enumerate}
          \item  Le travail de chacune des forces est nul 
          \item Le travail de la somme des forces est nul
          \item La somme des travaux de ces deux forces n’est pas nulle
      \end{enumerate}
  \item La fréquence f 
	  \begin{enumerate}
		  \item s’exprime en Hertz ( Hz )
		  \item est : f = $\frac{T}{2\pi}$
	  \end{enumerate}

  \item un disque tourne autour d’un axe fixe avec une vitesse de 600 tr/min ; la vitesse angulaire de ce disque est : 
  \begin{enumerate}
	  \item $\omega = 20\pi. rad/s$
	  \item $\omega = 10\pi. rad/s$
	  \item $\omega =4\pi. rad/s$
	  \item $\omega =\pi .rad/s$
  \end{enumerate}

  \vspace{1.5cm}
\item le travail $\omega_{AB}$ d’une force constante dont le point d’application M se déplace du point A au point B est donnée
	par la relation suivante : ($\alpha$ est l’angle entre $\vec{F}$ et$\vec{AB}$ )
  \begin{enumerate}
	  \item $W_{AB} =F.AB$
	  \item $W_{AB} =F.AB.cos(\alpha)$
	  \item $W_{AB} =F.AB.sin(\alpha)$
  \end{enumerate}
\item Le travail d’une force constante, lors du déplacement de son point d’application entre A et B.
 \begin{enumerate}
	 \item  ne dépend pas du chemin suivi entre A et B
	 \item dépend du chemin suivi entre A et B
  \end{enumerate}
\item La puissance instantanée $\mathscr{P}$ d’une force $\vec{F}$ est :
 \begin{enumerate}
	 \item $\mathscr{P} = \vec{F}.\vec{AB}$
	 \item $\mathscr{P} = \vec{F}.\vec{v}$
	 \item $\mathscr{P} = \vec{AB}.\vec{v}$
  \end{enumerate}
\item L’énergie cinétique EC d’un solide en mouvement de translation est définie par :
 \begin{enumerate}
	 \item $E_C = \frac{1}{2}.V.m^2$
	 \item $E_C = \frac{1}{2}.m.V^2$
	 \item $E_C = 2.m.V^2$
	 \item $E_C = 2.v.m^2$
	   \end{enumerate}
   \item Au voisinage de la Terre, l’énergie potentielle de pesanteur d’un solide de masse m ( si l’ axe Oz est vertical et orienté vers le haut, ) est définie par :
 \begin{enumerate}
	 \item $E_p = -mgz +C$
	 \item $E_p = mgz + C$
	 \item $E_p = \frac{1}{2}.mgz^2 + C$
	 \item $E_p = -\frac{1}{2}.mgz^2 + C$
\end{enumerate}
\item L’énergie potentielle Ep augmente lorsque l’altitude du solide:
 \begin{enumerate}
	 \item Augmente.
	 \item Diminue.
	 \item Reste Constante.
\end{enumerate}


%\vspace{2cm}


\item un corps S de masse $m = 2Kg$ et à l’altitude $h = 20 m$ , se déplace en chute libre d’un point A ( $z_A = 20 m$,$V_A$=$10 m.s^{-1}$) à un point B($z_B$=$10 m$ ) pendant une durée $\Delta{t} =  10 s$ . on donne $g =10N.Kg^{-1}$.
 \begin{enumerate}
	 \item[12.a] 	 L’expression littérale du travail du poids $\vec{P}$ du corps S est : 
		 \begin{enumerate}
			 \item  $W_{AB}(\vec{P}) = mg ( z_A – z_B )$		
			 \item  $W_{AB}(\vec{P}) = -mgh$
			 \item  $W_{AB}(\vec{P}) = mgh$

		 \end{enumerate}

	 \item[12.b] la valeur du travail du poids $\vec{P}$ est :
		 \begin{enumerate}
			 \item  $W_{AB}(\vec{P}) = -200J$		
			 \item  $W_{AB}(\vec{P}) = 200J$
			 \item  $W_{AB}(\vec{P}) = 20J$
			 \item  $W_{AB}(\vec{P}) = 2J$

		 \end{enumerate}

		 \vspace{5cm}
	 \item[12.c] La valeur de la puissance moyenne du poids $\vec{P}$ est :  
 \begin{enumerate}
	 \item $\mathscr{P_m} = 2W$
	 \item $\mathscr{P_m} = 20W$
	 \item $\mathscr{P_m} = 200W$
	 \item $\mathscr{P_m} = -200W$
  \end{enumerate}
\item[12.d] La valeur de l’énergie cinétique du corps (S) en point A est : 
	\begin{enumerate}
	 \item $E_CA = 200J$
	 \item $E_CA = 100J$
	 \item $E_CA = 50J$
	 \item $E_CA = -100J$
	   \end{enumerate}
   \item[12.e] La valeur de la variation de l’énergie cinétique du corps S entre A et B est :
\begin{enumerate}
	\item $\Delta{E_C} = 50J$
	\item $\Delta{E_C} = 100J$
	\item $\Delta{E_C} = 200J$
	\item $\Delta{E_C} = -200J$
	   \end{enumerate}


\end{enumerate}
\end{enumerate}
\end{multicols}

 %Parte 2 Electrodynamique : 
\vspace{-1cm}
\begin{multicols}{2}
    [
	\section*{\begin{center}\underline{Partie 2 :  électrodynamique }\end{center}}
    ]
	\begin{enumerate}
	\item La puissance électrique $\mathscr{P}_e$ reçue par un récepteur $AB$ pendant une durée $\Delta{t}$ est : 
			\begin{enumerate}
				\item $\mathscr{P}_e = U_{AB}.I.\Delta{t}$
			\item $\mathscr{P}_e = \frac{U_{AB}}{I}$
		\item $\mathscr{P}_e = U_{AB}.I$	
			\end{enumerate}
		\item L’énergie électrique dissipée $W_J$ par un conducteur ohmique $AB$ de résistance R pendant une durée s’écrit :
			\begin{enumerate}
				\item $W_J = U_{AB}.I.\Delta{t}$
				\item $W_J = R.I.\Delta{t}$
				\item $W_J = R.I^2.\Delta{t}$
				\item $W_J = \frac{U^2_{AB}}{R}.\Delta{t}$
			\end{enumerate}
		\item La tension $U_PN$ aux bornes d’un générateur débitant un courant électrique d’intensité I sortant par sa borne positive P, est donnée par la relation suivante :
			\begin{enumerate}
				\item $U_{PN} = E - r.I$
				\item $U_{PN} = E + r.I$
				\item $U_{PN} = EI - r.I^2$
			\end{enumerate}
		\item La puissance utile fournie par un récepteur ( E' , r' ) est :
			\begin{enumerate}
				\item $W_u=E'.I$
				\item $W_u=U_{AB}.I$
				\item $W_u=E'.I+  r'.I^2$
				\item $W_u=r'.I$
			\end{enumerate}
		\item Le rendement d’un récepteur est :
			\begin{enumerate}
				\item $\rho = \frac{W_J}{W_e}$
				\item $\rho = \frac{W_J}{W_u}$
				\item $\rho = \frac{W_u}{W_e}$
				\item $\rho = \frac{P_u}{W_e}$
			\end{enumerate}
		\item Les dispositifs suivants sont sources de champ magnétique :
			\begin{enumerate}
				\item Un fil de cuivre
				\item Un fil de cuivre parcouru par un courant
				\item La terre
				\item Un morceau de plastique frotté
				\item L’aiguille d’une boussole
			\end{enumerate}
		\item Les lignes de champ magnétique d’un aimant sortent par le pôle Sud et rentrent par le pôle Nord
		
			(a).Vrai \hspace{1cm} (b).Faux

\vspace{3cm}
		\item A l’intérieur d’un solénoïde long :
			\begin{enumerate}
				\item Le champ magnétique est uniforme, de vecteur $\vec{B}$ parallèle à l’axe du solénoïde
				\item Son sens , ne dépend pas du sens du courant
				\item Sa valeur (en tesla T) est donnée par l’expression : $B = \mu_0.\frac{N.I}{L}$ 
				\item Sa valeur en (T) est donnée par l’expression : $B = \mu_0.N.L$
			\end{enumerate}

	\end{enumerate}
\end{multicols}

%Partie #3 Optique 
\section*{\vspace{-2cm}\begin{center}\underline{Partie 3 :Optique}\end{center}}
	\begin{enumerate}
		\item Principe de propagation rectiligne de la lumière : la lumière se propage en ligne droite dans le vide et dans tout
milieu transparent et homogène
\begin{enumerate}
\item Vrai \item Faux
	\end{enumerate}
\item La relation caractéristique de la deuxième loi de Descartes de réfraction est :
	\begin{enumerate}
		\item $n_1.cos(i_1) = n_2.cos(i_2) $ \hspace{1cm}
		\item 	 $n_1.sin(i_1) = n_2.sin(i_2) $\hspace{1cm}
		\item	 $n_2.sin(i_1) = n_1.sin(i_2) $\hspace{1.1cm}
			
		\item  $n_1.cos(i_1) = n_2.sin(i_2) $
	\end{enumerate}
	\end{enumerate}
\begin{center}
	%\vspace{cm}
\hrulefill
\textbf{Chimie 30\%}
\hrulefill
\end{center}

\begin{multicols}{2}
    [
        \section*{Partie 4 :Chimie}
    ]
	\begin{enumerate}
		\item La quantité de matière d’un échantillon d’une espèce chimique X et de masse m ( X ) est donnée par la relation :
			\begin{enumerate}
				\item $n(X) = \frac{m(X)}{M(X)}$
				\item $m(X).M(X)$
				\item $n(X) = \frac{M(X)}{m(x)}$
			\end{enumerate}
		\item Pour un gaz parfait, la quantité de matière n , la température T , la pression P et le volume sont reliés par
l’équation du gaz parfait.
		\begin{enumerate}
			\item $P.T = n.R.T$
			\item $R.V = n.P.T$
			\item $P.V = n.R.T$
		\end{enumerate}
		\item La température absolue est donnée par la relation suivante :
			\begin{enumerate}
				\item $T(K)=T(^{\circ}C) + 273,15$
				\item $T(K)=T(^{\circ}C) - 273,15$
				\item $T(K)=T(^{\circ}C) + 273,25$
				\item $T(K)=T(^{\circ}C) - 273,25$
			\end{enumerate}
		\item La concentration molaire d’un soluté moléculaire X dissous dans une solution homogène est définie par :
			\begin{enumerate}
				\item $c(X) = n(X).V$
				\item $c(X) = m(X).V$
				\item $c(X) = \frac{m(X)}{V}$
				\item $c(X) = \frac{n(X)}{V}$
			\end{enumerate}
		\item Déterminer la concentration d’un soluté de quantité de matière $n ( x ) = 2mol$ dissoute dans un volume $V = 4L$.
			\begin{enumerate}
				\item $c(X) = 2 mol.L^{-1}$
				\item $c(X) = 0.5 mol.L^{-1}$
				\item $c(X) = 8 mol.L^{-1}$
				\end{enumerate}
			\item Une liaison entre deux atomes est polarisée si ces deux atomes sont: 
				\begin{enumerate}
					\item Différentes \item Identiques

			\end{enumerate}
			\item La conductance $G$ d’une portion de solution ionique , de section S et de longueur L , peut sous mettre sou la forme.
				\begin{enumerate}
					\item $G = \frac{S}{L}.\sigma$
					\item $G = S.L.\sigma$
					\item $G = \frac{L}{S}.\sigma$
				\end{enumerate}
			\item Calculer la conductivité , à 25 °C , d’une solution de nitrate de potassium     $(K^+_{(aq)} + {NO^-_3}_{(aq)})$ de concentration $C = 10 mol.m^{-3}$ . 

				On donne : 

				à 25 °C , $\lambda_{K^+}$=$7,35.10^{-3}S.m^2/mol$;  $\lambda_{{NO^-_3}_{(aq)}}$=$7,14 .10^{-3} S.m^2/mol$
				\begin{enumerate}
					\item $\sigma = 73,5Sm^{-1}$
					\item $\sigma = 7,14Sm^{-1}$
					\item $\sigma = 0,14Sm^{-1}$
				\end{enumerate}
			\item L’équation de la réaction ente l’aluminium et le soufre s’écrit : 
			\begin{enumerate}
				\item \ce{$3AL_{(s)}$ + $2S_{(s)}$ -> ${AL_2S_3}_{(s)}$}
				\item \ce{$2AL_{(s)}$ + $3S_{(s)}$ -> ${AL_2S_3}_{(s)}$}
				\item \ce{$6AL_{(s)}$ + $6S_{(s)}$ -> ${AL_2S_3}_{(s)}$}
				\item \ce{$2AL_{(s)}$ + $3S_{(s)}$ -> ${AL_2S_3}_{(s)}$}
			\end{enumerate}
		\item La réaction entre le fer solide et les ions ${H^+}_{(aq)}$ produit un dégagement de dihydrogène selon l’équation : \ce{$Fe_{(s)} + 2H^+_{(aq)}$ -> $Fe^{2+}_{(aq)}$ + ${H_2}_{(g)}$}

			Dans l’état initial : $n_i(Fe) = 9,0mmol$ ; $n_i(H^+) = 25,0mmol$ ;

			Dans les conditions de l’expérience, le volume molaire des gaz vaut : $V_m = 24,0Lmol^{-1}$
			\begin{enumerate}
				\item L’avancement maximal est : 
					\begin{enumerate}
						\item $X_{max} = 9,0mmol$
						\item $X_{max} = 12,5mmol$
						\item $X_{max} = 25,0mmol$
						\item $X_{max} = 24,0mmol$
					\end{enumerate}
				\item Le réactif limitant est :  
					\begin{enumerate}
						\item Le fer $Fe_{(s)}$
						\item L’ion $H^+_{(aq)}$
						\item L’ion $Fe^{2+}_{(aq)}$
					\item dihydrogène ${H_2}_{(g)}$
					\end{enumerate}
				\item Le volume de dihydrogène dégagé est : 
					\begin{enumerate}
						\item $V (H_2) = 300 mL$
						\item $V (H_2) = 216 mL$
						\item $V (H_2) = 600 mL$
						\item $V (H_2) =  .....  mL$
					\end{enumerate}
			\end{enumerate}
		\item La formule brute des alcanes est : 
			\begin{enumerate}
				\item $C_nH_{2n}$
				\item $C_nH_{2n+2}$
				\item $C_nH_{2n+1}$
				\item $C_{2n}H_{2n}$
			\end{enumerate}
		\item Le nom de la molécule suivante : 

			\chemfig{-[1](-[2])-[-1]-[1]OH }
			\begin{enumerate}
				\item 2-méthylpropane
				\item 2-méthylpropan-1- ol
				\item 2-méthylbutan- 2- ol
					\item Butanol
			\end{enumerate}
		\item Le groupe caractéristique des alcools est : 
			\begin{enumerate}
				\item \ce{-COOH}
				\item \ce{-CNH}
				\item \ce{-OH}
				\item \ce{-CHO}
			\end{enumerate}
	 \item  La concentration massique d'un soluté dans une solution s'exprime en :
	  \begin{enumerate}
		  \item $L.g^{-1}$
		  \item $g.L^{-1}$
		  \item $g$
		  \item $L$
	  \end{enumerate}

 \item  L'expression de la densité d'une espèce chimique X liquide ou solide vaut: 
	  \begin{enumerate}
		  \item $d = \rho_X \times \rho_{eau}$
		  \item $d = \frac{m_X }{m_{eau}}$
		  \item $d = \frac{\rho_{eau}}{\rho_{X}}$  
	  \end{enumerate}


	\end{enumerate}
\end{multicols}











%%%%%%%+_+_+_+_+_+_+_+_+_Partie2
%________________________________________
\end{document}
